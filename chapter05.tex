% !TEX encoding = UTF-8 Unicode
% -*- coding: UTF-8; -*-
% vim: set fenc=utf-8

\chapter{Abordagem para análise de rastros de proveniência}%
\label{chap:rastros-de-proveniencia}

% ilustrar com query SQL, tabela, grafo (img)

\section{Visão geral sobre a análise de rastros de proveniência}

 % definição, etc
 
\section{Tipos de rastros de proveniência}

% explicação + exemplo + complemento de uma (sub)seção anterior

\subsection{Físico}

\subsection{Lógico}

\subsection{Híbrido}

% \section{DfAnalyzer: Uma instanciação da arquitetura ARMFUL para análise dos rastros de proveniência}

\section{Query Processor}

O Query Processor foi implementado na linguagem de programação Java.

\perrotta{TODO: EXPAND}

\subsection{Detecção das últimas transformações de dados}

O \autoref{lst:algorithm-last-transformations} demonstra como obter as últimas transformações de dados de um fluxo de dados, isto é, as transformações de dados as quais não possuem nenhuma outra transformação de saída após as mesmas. A ideia é bastante simples: basta checar todas as dependências de dados \( \phi \) --- do conjunto de dependência de dados \( \Phi \) --- cujo \( dt_{\textrm{next}} \) é nulo, e tomar o \( dt_{\textrm{previous}} \) das mesmas.

\perrotta{TODO: complexidade, tipos de estruturas de dados, on-the-fly add na criação do dataflow}

% TODO: tipo de Phi: dependências de dados ou dataflow?
% TODO: tipo de transformations?

\begin{lstlisting}[language=pseudocode,label={lst:algorithm-last-transformations},caption={[Detecção das últimas transformações de dados]Detecção das útimas transformações de dados em uma especificação de fluxo de dados.}]
    function getLastTransformations(%\Phi%):
        transformations %\leftarrow% {}
        for each %\phi% %\in% %\Phi% do:
            if %$\phi.dt_{\texttt{next}} == \varnothing$% then:
                transformations %\leftarrow% transformations + 
            end if
        end for
        return transformations
    end function
\end{lstlisting}

% algoritmos, pré-processamento, exemplos ==> Java
% pseudocódigo
% complexidade, estruturas de dados

\subsection{Detecção das trilhas de transformações}

\perrotta{TODO: add autoref, caption repeating, basic explanation and complexity}
\perrotta{TODO: renew / recheck variable names, convention and style}

\begin{lstlisting}[language=pseudocode,label={lst:algorithm-transformation-tracks},caption={[Detecção das trilhas de transformações]Detecção do rastro em nível de fluxo de dados baseado nas trilhas de transformações.}]
    function getTransformationTracks(%\Phi%):
        lastDts %\leftarrow% getLastTransformations(%\Phi%)
        queue %\leftarrow% {}
        queue.push(lastDts)
        tracks %\leftarrow% {}
        while queue is not empty do:
            dt %\leftarrow% queue.pop()
            nextDts %\leftarrow% getNextTransformations(%\Phi%, dt)
            if lastDts.contains(dt) or hasManyOutputDatasets(%\Phi%, dt) or anyTransformationHasManyInputDatasets(nextDts, %\Phi%) then:
                TransformationTrack track %\leftarrow% TransformationTrack()
                track.addTransformation(dt)
                tracks %\leftarrow% tracks + {track}
            else then:
                track %\leftarrow% getTrack(tracks, nextDts)
                track.addTransformation(dt)
            end if
            previousDts %\leftarrow% getPreviousTransformations(%\Phi%, dt)
            queue.push(previousDts)
        end do
        return tracks
    end function
\end{lstlisting}

\silva{Implementação do processador de consultas}

\perrotta{Pré-processamento}
\perrotta{Pseudocódigos + explicações}
\perrotta{Como rodar}% explicar a assinatura da funcao principal