% !TEX encoding = UTF-8 Unicode
% -*- coding: UTF-8; -*-
% vim: set fenc=utf-8

\chapter{Abordagem para análise de rastros de proveniência}%
\label{chap:rastros-de-proveniencia}

% ilustrar com query SQL, tabela, grafo (img)

\section{Visão geral sobre a análise de rastros de proveniência}

 % definição, etc
 
\section{Tipos de rastros de proveniência}

% explicação + exemplo + complemento de uma (sub)seção anterior

\subsection{Físico}

\subsection{Lógico}

\subsection{Híbrido}

% \section{DfAnalyzer: Uma instanciação da arquitetura ARMFUL para análise dos rastros de proveniência}

\section{Query Processor}

O Query Processor foi implementado na linguagem de programação Java, a qual foi escolhida pelas seguintes razões:

\begin{itemize}
    \item possui um bom suporte a orientação a objetos;
    \item está bastante consolidada no mercado e na academia, sendo uma escolha tradicional;
    \item facilita a integração do Query Processor com os outros componentes do DfAnalyzer, já que eles também foram implementados em Java.
\end{itemize}

O projeto foi gerenciado e compilado pelo \texttt{Apache Maven 3.5.0}, tendo sido desenvolvido no ambiente de desenvolvimento integrado \texttt{NetBeans 8.2}, com a versão 1.8 do \texttt{JDK} do \texttt{Java}. O \textit{git} foi utilizado como sistema de controle de versão para o código, que consiste de:

\begin{itemize}
\item 22 arquivos-fonte \texttt{*.java};
\item 152 testes, utilizando o \textit{framework} JUnit.
\end{itemize}

\perrotta{TODO: EXPAND ++ TODO: Mencionar o pré-processamento e as otimizações que fiz}

\section{Algoritmos utilizados}%
\label{sec:algoritmos-utilizados}

Nessa seção serão abordados os principais algoritmos e funções utilizadas no Query Processor. Funções auxiliares referenciadas pelos mesmos estão definidas no \autoref{app:funcoes-auxiliares}. A notação utilizada é de pseudocódigo, e as definições dos algoritmos e nomes das variáveis utilizados foram ligeiramente alterados e simplificados em relação à implementação original, visando uma melhor clarificação e apresentação do código.

\perrotta{REVIEW: mencionar algo a mais sobre a notação?}

\subsection{Detecção das últimas transformações de dados}

O \autoref{lst:algorithm-last-transformations} demonstra como obter as \textbf{últimas transformações de dados} \texttt{transformations} de um fluxo de dados \( D \), isto é, as transformações de dados \(dt\) as quais não possuem nenhuma outra transformação de saída após as mesmas. A ideia do algoritmo é bastante simples: basta checar todas as dependências de dados \( \phi \) --- do conjunto de dependência de dados do fluxo de dados \( D.\Phi \) --- cujo \( dt_{\textrm{next}} \) é nulo, e tomar o \( dt_{\textrm{previous}} \) das mesmas.

% https://tex.stackexchange.com/questions/73231/avoid-page-breaks-in-lstlistings
\begin{minipage}[c]{0.95\textwidth}
\begin{lstlisting}[language=pseudocode,label={lst:algorithm-last-transformations},caption={[Detecção das últimas transformações de dados]Detecção das útimas transformações de dados em uma especificação de fluxo de dados.}]
function getLastTransformations(%\(D\)%):
    transformations %\leftarrow% {}
    for each %\phi% %\in% %\(D.\Phi\)% do:
        if %$\phi.dt_{\texttt{next}} = \varnothing$% then:
            transformations %\leftarrow% transformations + 
        end if
    end do
    return transformations
end function
\end{lstlisting}
\end{minipage}

A complexidade de tempo do algoritmo é linear da ordem de \( O(\#(D.\Phi)) \), isto é, proporcional ao número de dependências de dados do fluxo de dados \( D \), e as transformações retornadas pelo algoritmo são armazenadas em uma simples lista encadeada, já que não é necessário acesso aleatório a essa estrutura de dados. Entretanto, na prática, com o objetivo de amortizar um pouco a complexidade, esse algoritmo é aplicado \textit{on-the-fly}, no momento em que o fluxo de dados é construído e instanciado no QP (c.f. \autoref{subsec:preprocessamento}).

\subsection{Detecção das trilhas de transformações}%
\label{subsec:deteccao-das-trilhas-de-transformacoes}

O \autoref{lst:algorithm-transformation-tracks} demonstra como obter as \textbf{trilhas de transformações} \texttt{dtTracks} de um fluxo de dados \( D \). As trilhas são uma maneira natural de separar e agrupar transformações de dados em vários subconjuntos, definidos e limitados através de separações (do inglês, \textit{splits}) em \( D \). Tais separações existem no grafo do fluxo de dados em interseções de transformações que possuem mais de uma transformação de entrada ou de saída (\textit{i.e.}, grau do vértice maior do que 1). Por exemplo, na \autoref{fig:transformation-tracks}, podemos visualizar um exemplo de divisão e agrupamento de transformações de dados em trilhas.

\begin{figure}[htb]
    \centering
    \includegraphics[width=\textwidth]{img/transformation-tracks}
    \caption[Exemplo de detecção das trilhas de transformações.]{Exemplo de detecção das trilhas de transformações de um fluxo de dados. Na figura existem três trilhas de transformações (\textsc{TT}).}%
    \label{fig:transformation-tracks}
\end{figure}

O algoritmo começa com as últimas transformações \texttt{lastDts} de \( D \) (encontradas no \autoref{lst:algorithm-last-transformations}), e caminha em direção às primeiras transformações de \( D \), utilizando uma estrutura de dados \texttt{queue} do tipo fila (FIFO - \textit{First In First Out}) como armazenamento temporário das próximas transformações de dados a serem analisadas. A ideia principal é realizar a detecção do fim --- e, consequentemente, também do início --- de cada trilha de transformação \texttt{dtTrack}, que pode acontecer em várias situações: \textit{e.g.}, sempre que o grau de saída do vértide de uma transformação for maior do que 1, indicando que a transformação em questão possui várias transformações de saída.

\begin{minipage}[c]{0.95\textwidth}
\begin{lstlisting}[language=pseudocode,label={lst:algorithm-transformation-tracks},caption={[Detecção das trilhas de transformações]Detecção do rastro do fluxo de dados no nível de trilhas de transformações.}]
function getTransformationTracks(%\(D\)%):
    dtTracks %\leftarrow% {}
    lastDts %\leftarrow% getLastTransformations(%\(D\)%)%~\quad%#%~\autoref{lst:algorithm-last-transformations}%
    queue %\leftarrow% {}%~\quad%#%~%FIFO (First In First Out)
    queue.enqueue(lastDts)
    while queue is not empty do:
        dt %\leftarrow% queue.dequeue()
        nextDts %\leftarrow% getNextTransformations(%\(D\)%, dt)
        if dt %\in% lastDts
           or hasManyOutputDatasets(%\(D\)%, dt)
           or hasManyNextTransformations(%\( D \)%, dt)
           or anyTransformationHasManyInputDatasets(%\(D\)%, nextDts)
           then:
            dtTrack %\leftarrow% {dt}
            dtTracks %\leftarrow% dtTracks + {dtTrack}
        else then:
            dtTrack %\leftarrow% getTransformationTrack(dtTracks, nextDts[0])
            dtTrack %\leftarrow% dtTrack + {dt}
        end if
        previousDts %\leftarrow% getPreviousTransformations(%\(D\)%, dt)
        queue.enqueue(previousDts)
    end do
    return dtTracks
end function
\end{lstlisting}
\end{minipage}

A complexidade de tempo do algoritmo é linear da ordem de \( O(\#(D.\Phi)) \), como no \autoref{lst:algorithm-last-transformations}. As trilhas de transformações \texttt{dtTracks} são armazenadas em listas encadeadas de transformações --- acesso aleatório não é necessário, porém convém armazená-las na mesma ordem em que foram encontradas.

\subsection{Detecção das trilhas de conjuntos de dados}

O \autoref{lst:algorithm-dataset-tracks} tem como função obter as \textbf{trilhas de conjuntos de dados} \texttt{dsTracks} de um fluxo de dados \( D \). O conceito de trilha de conjuntos de dados é análogo e baseado no de trilha de transformações de dados, mencionado na \autoref{subsec:deteccao-das-trilhas-de-transformacoes}: é uma forma de dividir e agrupar conjuntos de dados em diversos subconjuntos.

O algoritmo funciona com base no \autoref{lst:algorithm-transformation-tracks}: uma vez tomadas as trilhas de transformações de dados \texttt{dtTracks}, é simples obter cada uma das trilhas de conjuntos de dados \texttt{dsTrack} a partir de cada \texttt{dtTrack}. Para isso, basta tomar a união de todos os conjuntos de dados adjacentes (\textit{i.e.} anteriores e posteriores) a todas as transformações de dados de \texttt{dtTrack}.

\begin{minipage}[c]{0.95\textwidth}
\begin{lstlisting}[language=pseudocode,label={lst:algorithm-dataset-tracks},caption={[Detecção das trilhas de conjuntos de dados]Detecção do rastro do fluxo de dados no nível de trilhas de conjuntos de dados.}]
function getDatasetTracks(%\(D\)%):
    dsTracks %\leftarrow% {}
    dtTracks %\leftarrow% getTransformationTracks(D)%~\quad%#%~\autoref{lst:algorithm-transformation-tracks}%
    for each dtTrack %\in% dtTracks do:
        dsTrack %\leftarrow% {}
        for each dt %\in% dtTrack do:
            if dsTrack is empty then:
                dsTrack %\leftarrow% dsTrack + {getNextDatasets(%\(D\)%, dt)}
            end if
            dsTrack %\leftarrow% dsTrack + {getPreviousDatasets(%\(D\)%, dt)}
        end do
        dsTracks %\leftarrow% dsTracks + {dsTrack}
    end do
    return dsTracks
end function
\end{lstlisting}
\end{minipage}

A complexidade de tempo do algoritmo é a mesma do algoritmo do \autoref{lst:algorithm-transformation-tracks}: linear da ordem de \( O(\#(D.\Phi)) \), e as trilhas de conjuntos de dados \texttt{dsTracks} são também armazenadas em listas encadeadas de conjuntos de dados.

\subsection{Obtenção de múltiplos mapeamentos de atributos entre dois conjuntos de dados}

O \autoref{lst:algorithm-attribute-mappings} demonstra como obter \textbf{múltiplos mapeamentos de atributos}...

\perrotta{TODO: introduction with caption repeating + algorithm explanation. Reiterate the definition of mapeamentos de atributos (seção 2, das definições).}

\begin{minipage}[c]{0.95\textwidth}
\begin{lstlisting}[language=pseudocode,label={lst:algorithm-attribute-mappings},caption={[Obtenção de múltiplos mapeamentos de atributos]Obtenção de múltiplos mapeamentos de atributos entre dois conjuntos de dados adjacentes.}]
function getAttributesMapping(dt,%\(ds_{\textrm{input}}\)%,%\(ds_{\textrm{output}}\)%,type)
   attrs %\leftarrow% {}
   if type = physical then:
       attrs %\leftarrow% attrs + {t.getInstanceAttribute()}
   else:
       for each inAttr %\in% %\(ds_{\textrm{input}}\).A% do:
           for each outAttr %\in% %\(ds_{\textrm{output}}\).A% do:
               if inAttr.name = outAttr.name
                  and inAttr.type = outAttr.type then:
                   attrs %\leftarrow% attrs + {inAttr}
               end if
           end do
       end do
   end if
   return attributesMapping(attrs,dt,%\(ds_{\textrm{input}}\)%,%\(ds_{\textrm{output}}\)%)
end function
\end{lstlisting}
\end{minipage}

\perrotta{TODO: complexity and data structures}

\subsection{Obtenção dos caminhos entre dois conjuntos de dados}

A função \texttt{findPaths} do \autoref{lst:algorithm-find-paths} inicializa a estrutura de estados \texttt{state}, necessária para a chamada da função de busca em profundidade \abbrev{DFS}{Depth-first Search (Busca em profundidade)} \texttt{depthFirstSearch} (DFS) do \autoref{lst:algorithm-dfs}. Essas duas funções atuam em conjunto com um objetivo em comum: \textbf{obter todos os caminhos entre dois conjuntos de dados} \texttt{dsOrigin} e \texttt{dsDestination} do fluxo de dados \(D\).
Nesse contexto, um caminho é definido como uma lista ordenada de conjuntos de dados, cujo primeiro elemento é \texttt{dsOrigin}, e cujo último elemento é \texttt{dsDestination}.

Como podem existir potencialmente vários caminhos entre esse par de conjuntos de dados, a função \texttt{findPaths} possui dois argumentos adicionais que atuam como filtros:

\begin{itemize}
    \item \texttt{dsIncludes} --- uma coleção de conjuntos de dados: todos eles devem estar presentes no caminho para que tal caminho seja retornado pela função \texttt{findPaths};
    \item \texttt{dsExcludes} --- o oposto de \texttt{dsIncludes}: nenhum dos caminhos que fazem parte da coleção \texttt{dsExcludes} pode estar presente nos caminhos retornados por \texttt{findPaths}.
\end{itemize}

\begin{minipage}[c]{0.95\textwidth}
\begin{lstlisting}[language=pseudocode,label={lst:algorithm-find-paths},caption={[Obtenção dos caminhos entre dois conjuntos de dados.]Obtenção de todos os caminhos entre dois conjuntos de dados, inicializando a \textsc{DFS} com um estado inicial apropriado.}]
function findPaths(D,dsOrigin,dsDestination,dsIncludes,dsExcludes)
    state %\leftarrow% {}
    state.visited %\leftarrow% {dsOrigin}
    state.paths %\leftarrow% {}
    state.dsOrigin = dsOrigin
    state.dsDestination = dsDestination
    state.dsIncludes = dsIncludes
    state.dsExcludes = dsExcludes
    depthFirstSearch(D, state)%~\quad%#%~\autoref{lst:algorithm-dfs}%
    return state.paths
end function
\end{lstlisting}
\end{minipage}

\begin{minipage}[c]{0.95\textwidth}
\begin{lstlisting}[language=pseudocode,label={lst:algorithm-dfs},caption={[Busca em Profundidade (DFS) dos caminhos]Depth First Search (DFS): busca em profundidade de todos os caminhos entre dois conjuntos de dados.}]
function depthFirstSearch(D,state)
    nodes %\leftarrow% getNextDatasets(D,state.visited.getLast())
    for each node %\in% nodes do:
        if node %\in% state.visited then:
            continue
        end if
        if node = state.dsDestination then:
            state.visited %\leftarrow% state.visited + {node}
            path %\leftarrow% new Path(state.visited)
            if state.dsIncludes %\in% path and state.dsExcludes%~%%\not\in%%~%path then:
                state.paths %\leftarrow% state.paths + {path}
            end if
            state.visited.pop()
            break
        end if
    end do
    for node %\in% nodes do:
        if node %\in% state.visited or node = state.dsDestination then:
            continue
        end if
        state.visited %\leftarrow% state.visited + {node}
        depthFirstSearch(state)
        state.visited.pop()
    end do
end function
\end{lstlisting}
\end{minipage}

A função \texttt{depthFirstSearch} é responsável por realizar uma busca no grafo do fluxo de dados \(D\), tratando cada conjunto de dados como um vértice do grafo. Dois vértices são considerados adjacentes se existe uma transformação de dados que consome um dos vértices e produz o outro. A DFS opera recursivamente, utilizando a estrutura \texttt{state} para salvar dados intermediários durante a busca, e também para popular a coleção \texttt{state.paths} com os caminhos encontrados. A coleção de conjuntos de dados \texttt{state.visited} armazena os conjuntos de dados que já foram processados, a fim de evitar que a busca seja executada indefinidamente.

A complexidade computacional desse par de funções (\texttt{findPaths} e \texttt{depthFirstSearch}) é da ordem de \(O(\#(S) + \#(T))\), que é a complexidade computacional da DFS. Os caminhos encontrados são armazenados em uma lista encadeada.

\subsection{Geração da consulta em SQL}

\perrotta{TODO: everything. Autoref code label, Bold title, Introduction with caption repeating + algorithm explanation. Later: Complexity, data structures.}

\begin{minipage}[c]{0.95\textwidth}
\begin{lstlisting}[language=pseudocode,label={lst:algorithm-generate-sql-query},caption={[Geração da consulta em SQL]Geração da consulta na linguagem SQL a partir das especificações do usuário.}]
function generateSqlQuery(D,dsOrigins,dsDestinations,dsIncludes, dsExcludes,type,projections,selections)
    selectClause %\leftarrow% {}
    fromClause   %\leftarrow% {}
    whereClause  %\leftarrow% {}
    for each projection %\in% projections do:
        selectClause %\leftarrow% selectClause + {projection}
    end do
    for each selection %\in% selections do:
        whereClause %\leftarrow% whereClause + {selection}
    end do
    for each dsOrigin %\in% dsOrigins do:
        for each dsDestination %\in% dsDestinations do:
            paths %\leftarrow% findPaths(dsOrigin, dsDestination, dsIncludes, dsExcludes)%~\quad%#%~\autoref{lst:algorithm-find-paths}%
            for each path %\in% paths do:
                for each ds %\in% path do:
                    fromClause %\leftarrow% fromClause + {ds}
                end do                
                for (i %\leftarrow% 1; i < path.size(); i %\leftarrow% i + 1) do:
                    prevDs %\leftarrow% path[i - 1]
                    nextDs %\leftarrow% path[i]
                    attrMapping %\leftarrow% getAttributesMapping(D.getTransformation(prevDs, nextDs), prevDs, nextDs, type)%~\quad%#%~\autoref{lst:algorithm-attribute-mappings}%
                    for each attr %\in% attrMapping do:
                        whereClause.addSelection(prevDs.attr, "=", nextDs.attr)
                        if projections is empty then:
                            selectClause %\leftarrow% selectClause + {prevDs.attr, rightDs.attr}
                        end if
                    end do
                end do
            end do
        end do
    end do
    sqlQuery %\leftarrow% new SqlQuery(selectClause, fromClause, whereClause)
    return sqlQuery
end function
\end{lstlisting}
\end{minipage}

\subsection{Pré-processamento}%
\label{subsec:preprocessamento}

\section{Um exemplo}

\perrotta{como rodar o algoritmo -- explicar a assinatura da função principal e exemplos de uso (chamadas)}