% !TEX encoding = UTF-8 Unicode
% -*- coding: UTF-8; -*-
% vim: set fenc=utf-8

\chapter{Conclusão}%
\label{chap:conclusao}

\section{Conclusão}

Nessa monografia foi apresentada a implementação de um Processador de Consultas (\textit{Query Processor}) como um componente integrante ao DfAnalyzer, instanciando a arquitetura ARMFUL para a análise de simulações computacionais em larga escala.

\perrotta{TODO: EXPAND}

\section{Trabalhos Futuros}

Dentre algumas ideias de trabalhos futuros para o projeto do processador de consultas, destaca-se:

\begin{itemize}
    % Projeto da Débora: visualizador
    \item a implementação e integração de uma \textbf{interface gráfica} para o processador de consultas, com o intuito de criar uma boa experiência de usabilidade para o usuário e, mais especificamente: (\(i\)) auxiliá-lo a criar e especificar consultas, através da ação de arrastar e soltar com o \textit{mouse} etiquetas com as transformações e conjuntos de dados; e (\(ii\)) apresentá-lo os resultados da execução da consulta em SQL no banco de dados de forma visualmente organizada e limpa.
    % Otimizações de consultas que discutimos.
    \item aumentar a \textbf{performance das consultas} em SQL, com uma (\(i\)) diminuição do \textit{overhead} do processador de consultas, através do aumento do desempenho da conversão das especificações do usuário para a linguagem SQL; e da (\(ii\)) otimização da consulta em si, por exemplo, através de melhorias nas técnicas de indexação, ou da diminuição do número de projeções incluídos na mesma, ou mesmo via um melhor aproveitamento das informações e dados de proveniência disponíveis no banco de dados.
    \item implementar uma forma de \textbf{validação nas especificações do usuário}, permitindo apenas a geração de consultas sintática e semanticamente válidas, com o intuito de melhorar o \textit{design} de interação e diminuir a probabilidade de erros por parte do mesmo, \textit{e.g.} permitir apenas conjuntos de dados existentes nas projeções especificadas pelo usuário.
\end{itemize}

\section{Considerações Finais}

\perrotta{Capítulo 07 --- TODO: Considerações Finais. Depende dos resultados dos experimentos.}
