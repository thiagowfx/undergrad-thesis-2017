% !TEX encoding = UTF-8 Unicode
% -*- coding: UTF-8; -*-
% vim: set fenc=utf-8

% REVIEW: {referencial, embasamento} teórico
\chapter{Referencial Teórico}%
\label{chap:referencial-teorico}

\perrotta{Capítulo 02 --- Referencial Teórico}

Nessa seção serão abordados alguns conceitos e a terminologia que será utilizada ao longo desse trabalho.

\section{Conceitos básicos}

\section{Especificações}

\subsection{Dataflow}

Um \textit{dataflow}%
\footnote{Em tradução livre, ``fluxo de dados''. No entanto, daqui em diante continuaremos a utilizar \textit{dataflow}, por ser um termo mais preciso.}
$D_F$ é definido como:
%
$$ D_F = (T, S, \Phi) $$
%
onde:
%
\begin{itemize}
    \item $T = \{t_1, t_2, \cdots, t_{\alpha}\}$ \vdots{} transformações
    \item $S = \{s_1, s_2, \cdots, t_{\beta}\}$ \vdots{} conjuntos de dados
    \item $\Phi = \{\phi_1, \phi_2, \cdots, \phi_{\gamma}\}$ \vdots{} dependências de dados
\end{itemize}

\subsection{Dependência de dados}

\subsection{Conjunto de dados}

\subsection{Atributo}

\subsection{Coleção de dados}

\subsection{Item de dados}

\subsection{Elemento de dados}
