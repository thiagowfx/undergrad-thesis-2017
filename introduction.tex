% !TEX encoding = UTF-8 Unicode
% -*- coding: UTF-8; -*-
% vim: set fenc=utf-8

\chapter{Introdução}

\section{Visão geral e Contextualização}

% Simulações computacionais...análises...hipóteses científicas...
Simulações computacionais tornaram-se predominantes e omnipresentes no dia-a-dia de cientistas\footnote{Daqui em diante denominados \textit{usuários}.}, sendo necessárias na realização de análises que utilizam modelos computacionais complexos que lidam com grandes volumes de dados~\cite{silva2015analyzing}, permitindo a exploração de dados específicos de domínio para apoiar esses usuários na validação de hipóteses científicas ou comportamentos peculiares.

% Muitos programas...consumem e produzem muitos dados...
Essas simulações geralmente envolvem a execução de muitos programas no domínio da aplicação os quais intensivamente consomem e produzem muitos dados, a maior parte dos quais é armazenada em uma miscelânea de formatos de arquivo também no domínio da aplicação~\cite{silva2015analyzing}; por exemplo, \abbrev{FITS}{Flexible Image Transport System} FITS~\cite{greisen2002representations} em astronomia, ou \abbrev{HDF}{Hierarchical Data Format} HDF5~\cite{hdfgroup2014hdf5} e \abbrev{NetCDF}{Network Common Data Form} NetCDF~\cite{rew1990netcdf} em dinâmica de fluidos computacionais.

% Longo tempo para ser executada...mesmo em CAD
Uma grande parte dessas simulações computacionais de larga escala leva um longo tempo para ser executada, mesmo em ambientes de \abbrev{CAD}{Computação de Alto Desempenho} CAD (Computação de Alto Desempenho)~\cite{silva2017raw}, onde o poder de computação é amplificado através da utilização de diversos núcleos computacionais.

% Rastrear grandezas...somente no fim...programas ad-hoc...SGBD
Durante as mesmas, usuários comumente precisam rastrear grandezas de interesse, tais como resíduos, tempo de execução e estimativas de erro, baseadas em elementos de dados relacionados de vários arquivos, a fim de controlar a sua execução ao máximo possível~\cite{silva2016situ}.
Entretanto, esse rastreio costuma ser realizado somente ao fim da simulação, o que \textit{a priori} não é prático para o usuário, já que essas simulações levam um tempo considerável de execução mesmo em CAD, como mencionado anteriormente~\cite{silva2017raw}; além disso, mesmo considerando que os arquivos produzidos sejam tipicamente apoiados por diversas linguagens de programação e bibliotecas, programas \textit{ad-hoc} precisam comumente ser desenvolvidos a fim de fazer análises em grande escala, as quais se tornam ainda mais custosas se realizadas com \abbrev{SGBD}{Sistema de Gerenciamento de Banco de Dados} SGBDs (Sistemas de Gerenciamento de Banco de Dados), pois eles aumentam a complexidade de execução em termos de estruturação de dados e carregamento na memória~\cite{silva2015analyzing}.

\section{Motivação}

% SGWfCs...paralelização...proveniência...
Para amenizar o custo de tempo de execução, as simulações computacionais citadas anteriormente podem ser gerenciadas por um \abbrev{SGWfC}{Sistema de Gerência de \textit{Workflows} Científicos} SGWfC (Sistema de Gerência de \textit{Workflows} Científicos) paralelo, tal como o Kepler~\cite{ludascher2006scientific} e o Pegasus~\cite{deelman2005pegasus}, o que faz com que se beneficiem da proveniência e do paralelismo de dados entre os diferentes programas que compõem o seu \textit{workflow}\footnote{Em tradução livre, ``fluxo de trabalho''. No entanto, daqui em diante continuaremos a utilizar \textit{workflow}, por ser um termo mais preciso.}~\cite{bux2013parallelization}.

% Workflows...dataflows(dt, ds, phi)...DAG...
Tais \textit{workflows} científicos são composições de tarefas de processamento de dados sequenciais e concorrentes, cuja ordem de execução é determinada com base nas interdependências entre os dados~\cite{bux2013parallelization}. Eles podem ser modelados através de um fluxo de dados (\textit{dataflow}) que descreve todos os conjuntos de dados, atributos, transformações e dependências entre os mesmos~\cite{silva2017raw}. Esses fluxos de dados são usualmente especificados na forma de um \abbrev{DAG}{Directed Acyclic Graph --- Grafo Direcionado Acíclico} DAG (grafo direcionado acíclico), em que tarefas (transformações) individuais são representadas como nós, e conjuntos de dados são modelados como arestas direcionadas.

Os SGWfCs citados anteriormente são usualmente difíceis de serem configurados e utilizados pelos usuários; alguns deles tais como o Taverna possuem uma boa ênfase em usabilidade, porém não compensam na paralelização e utilização de recursos computacionais distribuídos; perdendo, assim, sua efetividade.

\missingfigure{Exemplo de -> o ->, i.e., transformação com dataset de entrada e de saída}

Os SGWfCs facilitam o \textit{design}, a manutenção, a execução e o monitoramento de processamento de dados científicos

\section{Contribuição} % {Objetivos, Relevância, Método}

Para preencher esse gargalo, \perrotta[...]{ARMFUL}

\perrotta{Falar sobre a challenge disso, tipos de queries, etc}

\perrotta{Finalmente, introduzir o QueryProcessor}

\perrotta{Mencionar a arquitetura ARMFUL}

\perrotta{Mencionar a minha contribuição}

% TODO: expandir SWfMSs ++ citar minha contribuição (a. facilitar a query; b. desempenho (pré-processar); c. fazer a query sem saber escrever ela) 

% REVIEW: alt: {estruturação,organização} do {documento,trabalho,texto}
\section{Estruturação do Texto}

Este trabalho está organizado da seguinte forma:
o \autoref{chap:referencial-teorico} apresenta o embasamento teórico, introduzindo e definindo os conceitos e a terminologia que será utilizada ao longo do trabalho.
No \autoref{chap:trabalhos-relacionados} listamos algumas publicações relacionadas e recentes, algumas dentre as quais serviram como motivação e base para essa.
O \autoref{chap:arquitetura-armful} aborda a arquitetura \abbrev{ARMFUL}{Analysis of Raw Data from Multiple Files} ARMFUL (Analysis of Raw Data from Multiple Files), da qual a aplicação desenvolvida nesse trabalho é um componente.
No \autoref{chap:rastros-de-proveniencia} desenvolvemos a abordagem utilizada para analisar rastros de proveniência dos arquivos e dados de simulações científicas.
No \autoref{chap:experimentos} são abordados os resultados obtidos através do \textit{QueryProcessor} e os experimentos realizados com o mesmo.
Finalmente, o \autoref{chap:conclusao} conclui o trabalho com perspectivas de trabalhos futuros.
