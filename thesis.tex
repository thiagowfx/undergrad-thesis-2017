% !TEX encoding = UTF-8 Unicode
% -*- coding: UTF-8; -*-
% vim: set fenc=utf-8
\documentclass[grad,numbers]{coppetex/coppe}
\usepackage[utf8]{inputenc}
\usepackage{amsmath,amssymb}
\usepackage{lmodern}
\usepackage{float}
\usepackage{multirow}
\usepackage{subfigure}
\usepackage{listings}
\usepackage{longtable}
\usepackage{placeins}
\usepackage{hyperref}
\usepackage{todonotes}

% Configure listings package
% \lstset{ %
% language=Java,                % choose the language of the code
% basicstyle=\footnotesize\sf,    % the size of the fonts that are used for the code
% numbers=left,                   % where to put the line-numbers
% numberstyle=\footnotesize,      % the size of the fonts that are used for the line-numbers
% stepnumber=1,                   % the step between two line-numbers. If it's 1 each line 
%                                 % will be numbered
% numbersep=5pt,                  % how far the line-numbers are from the code
% showspaces=false,               % show spaces adding particular underscores
% showstringspaces=false,         % underline spaces within strings
% showtabs=false,                 % show tabs within strings adding particular underscores
% frame=single,                   % adds a frame around the code
% tabsize=4,                      % sets default tabsize to 4 spaces
% captionpos=t,                   % sets the caption-position to top
% breaklines=true,                % sets automatic line breaking
% breakatwhitespace=true,         % sets if automatic breaks should only happen at whitespace
% title=\lstname,                 % show the filename of files included with \lstinputlisting;
%                                 % also try caption instead of title
% escapeinside={\%*}{*)},         % if you want to add a comment within your code
% morekeywords={}                 % if you want to add more keywords to the set
% }
% \renewcommand{\lstlistingname}{Arquivo}

\makelosymbols{}
\makeloabbreviations{}

\begin{document}
  \title{Um novo método de proveniência de dados}
  \foreigntitle{A new method for data provenance}

  \author{Thiago}{Barroso Perrotta}
  \advisor{Prof.}{Marta}{Mattoso}{D.Sc.}
  \advisor{}{Vitor}{Silva}{D.Sc. Candidate}

  %\examiner{Prof.}{José Ferreira de Rezende}{Dr.}
  %\examiner{Prof.}{Aloysio de Castro Pinto Pedroza}{Dr.}

  \department{PESC}
  \date{06}{2017}

  \keyword{Bancos de Dados}
  \keyword{Java}

  \foreignkeyword{Databases}
  \foreignkeyword{Java}

  \listoftodos{}

  \maketitle
  \frontmatter{}

  \dedication{Dedico esse trabalho aos meus pais}
  % !TEX encoding = UTF-8 Unicode
% -*- coding: UTF-8; -*-
% vim: set fenc=utf-8

\dedication{Dedico esse trabalho aos meus pais, que sempre me apoiaram em tudo o que realizei.}

\chapter*{Agradecimentos}

Agradeço a minha mãe e ao meu pai, por todo o carinho e apoio que recebi, e por sempre permanecerem presentes ao meu lado tanto nas fases boas quanto nas ruins.

Agradeço ao Vitor Silva e à professora Marta Mattoso, meus orientadores, por todo o inestimável suporte e atenção nas diversas reuniões de projeto que tivemos.

Agradeço à professora Márcia Cerioli e ao professor Ricardo Marroquim, por toda a experiência transmitida durante meu período de realização de iniciação científica durante a graduação.

Agradeço ao meu amigo Bruno Buss pelo voto de confiança e pela serendipidade, mesmo apesar da distância.

Agradeço ao meu \textit{ex-host} na Google, Andy Venikov, por todos os ensinamentos e motivação.

Por fim, agradeço aos meus bons amigos, os quais dia a dia fazem-me crescer e tornar-me uma pessoa melhor.
  % !TEX encoding = UTF-8 Unicode
% -*- coding: UTF-8; -*-
% vim: set fenc=utf-8

\begin{abstract}

%REVIEW: resumo em português
Simulações computacionais frequentemente consomem e produzem um grande volume de dados. Parte desses dados é armazenado em arquivos brutos %silva2015analyzing
, que podem assumir diversos formatos, de acordo com o domínio da aplicação, %silva2017raw
tal como o \textit{HDF5} em simulações de dinâmica de fluidos computacionais. %silva2017raw
Durante essas simulações, usuários em geral precisam rastrear grandezas de interesse (como resíduos, tempo de execução e estimativas de erros) %silva2016situ
com base em elementos de dados relacionados de diversos arquivos, a fim de controlar sua execução ao máximo possível %silva2016situ.
No entanto, esse rastreio é realizado usualmente somente ao fim da simulação. %silva2016situ
Apesar dos formatos comuns de arquivo serem apoiados por diversas linguagens de programação e bibliotecas,
os usuários comumente precisam desenvolver programas \textit{ad-hoc} para a análise em grande escala dos mesmos %silva2015analyzing,
que pode ser ainda mais custoso se realizado com bancos de dados, pois eles exigem que os dados \silva[brutos]{ao invés de bruto, utilizar dado científico}
sejam estruturados e carregados em memória. %silva2015analyzing
Nesse cenário, Sistemas de Gerência de Workflows Científicos (\textit{SGWfC}) com ciência do fluxo de dados têm empregado a abstração de workflows científicos
para apoiar a execução paralela de simulações computacionais, em que os dados de proveniência podem favorecer a gerência de elementos de dados
relacionados em múltiplos arquivos de dados científicos (ou seja, produzidos por diferentes programas de simulação) %silva2015analyzing.
Esse trabalho utiliza a arquitetura de componentes ARMFUL, que utiliza uma abordagem baseada em dados de proveniência para extrair e
relacionar grandezas de interesse. Mais especificamente, esse trabalho contribui com um mecanismo de processamento de consultas durante a execução de simulações computacionais,
considerando-se o fluxo de elementos de dados ao longo das simulações computacionais.

\end{abstract}

\begin{foreignabstract}

%REVIEW: resumo em inglês
\silva[Computer simulations potentially receive and produce lots of raw data files.
These files usually follow a \textit{de facto} standard data format established by the application domain,
such as HDF5 in fluid dynamics simulations.
In these simulations, users often need to track quantities of interest (residuals, execution times, error estimates) in order to control as much as possible its execution.
However, this tracking is usually done only once the simulation ends.
This work presents a solution based on provenance data with the purpose of online (\textit{in situ}) extraction and connection among quantities of interest.]{traduzir novamente}

\end{foreignabstract}


  \tableofcontents
  \listoffigures
  \listoftables
  \printlosymbols{}
  \printloabbreviations{}

  \mainmatter{}
  \chapter{Introdução}

O protocolo HTTP (Hypertext Transfer Protocol).


  \backmatter{}
  \bibliographystyle{coppetex/coppe-unsrt}
  \bibliography{thesis}

  % \appendix

  % \chapter{Apendice capitulo}

  % Apendice capitulo.

\end{document}
