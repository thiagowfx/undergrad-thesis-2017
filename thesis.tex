% !TEX encoding = UTF-8 Unicode
% -*- coding: UTF-8; -*-
% vim: set fenc=utf-8

% use coppetex style class
\documentclass[grad,numbers,final]{coppetex/coppe}

% https://tex.stackexchange.com/questions/331430/produce-a-monochrome-pure-black-and-white-pdf-using-xelatex
% \usepackage[monochrome]{xcolor}

\usepackage{iftex}
% https://tex.stackexchange.com/questions/47576/combining-ifxetex-and-ifluatex-with-the-logical-or-operation
\newif\ifxetexorluatex
\ifxetex
  \xetexorluatextrue
\else
  \ifluatex
    \xetexorluatextrue
  \else
    \xetexorluatexfalse
  \fi
\fi

\ifPDFTeX
  % use UTF-8 in pdftex (not needed in LuaTeX and XeTeX)
  \usepackage[utf8]{inputenc}
\fi

% math and symbols
\usepackage{amsmath,amssymb}

 % next-gen computer modern fonts
\usepackage{lmodern}

% better placement of images in subsections
\usepackage[section]{placeins}

% https://en.wikibooks.org/wiki/LaTeX/Floats,_Figures_and_Captions
\usepackage{float}

% figures with borders
\floatstyle{boxed}
\restylefloat{figure}

% required for better figures
\usepackage{subfigure}

% required for better tables
\usepackage{longtable}
\usepackage{multirow}

% curly brackets besides tables (\ldelim, \rdelim)
\usepackage{bigdelim}

% quotes
\usepackage{epigraph}
\setlength{\epigraphrule}{0pt}

% text highlighting, underlining, spacing
\usepackage{soul}

% todo notes
\usepackage[obeyFinal,portuguese,colorinlistoftodos,textsize=footnotesize]{todonotes}
\newcommand{\mattoso}[2][]{\sethlcolor{pink}\texthl{#1}\todo[author=\textbf{Marta Mattoso},inline,color=pink]{#2}}
\newcommand{\perrotta}[2][]{\sethlcolor{orange}\texthl{#1}\todo[author=\textbf{Thiago Perrotta},inline,color=orange]{#2}}
\newcommand{\silva}[2][]{\sethlcolor{lightgray}\texthl{#1}\todo[author=\textbf{Vitor Silva},inline,color=lightgray]{#2}}

% captions for floats
\usepackage[labelfont=bf]{caption}

% source code listings
\usepackage{listings}

% https://www.overleaf.com/help/193-what-otf-slash-ttf-fonts-are-supported-via-fontspec
% https://www.overleaf.com/help/73-i-have-a-custom-font-id-like-to-load-to-my-document-how-can-i-do-this
\ifxetexorluatex
  \usepackage{fontspec}
  \newfontfamily{\lstsansserif}[Scale=.80]{Fira Mono}
  \lstset{basicstyle=\lstsansserif}
\else
  \lstset{basicstyle=\footnotesize\sf}
\fi

\lstset{%
  aboveskip=15pt,
  belowskip=15pt,
  breakatwhitespace=false,     % sets if automatic breaks should only happen at whitespace
  breaklines=true,             % sets automatic line breaking
  postbreak=\mbox{\textcolor{red}{$\hookrightarrow$}\space},
  captionpos=t,                % sets the caption-position
  columns=fullflexible,
  commentstyle=\color{gray},   % sets the comments color
  escapechar=\%,               % write LaTeX inside listings
  frame=lines,                 % adds a frame
  numbers=left,                % where to put the line-numbers
  numberstyle=\footnotesize\color{gray}, % the size of the fonts that are used for the line-numbers (alt: \tiny)
  showspaces=false,            % show spaces adding particular underscores
  showstringspaces=false,      % underline spaces within strings
  showtabs=false,              % show tabs within strings adding particular underscores
  stepnumber=1,                % the step between two line-numbers. If it's 1 each line
  stringstyle=\color{gray},
  tabsize=2,                   % sets default tabsize to 4 spaces
  title=\lstname,              % show the filename of files included with \lstinputlisting;
}

\lstdefinelanguage{sqlresults}{%
  aboveskip=5pt,
  belowskip=5pt,
  basicstyle=\footnotesize\ttfamily,
  columns=fixed,
  numbers=none,
  frame=none,
}

% highlight the following keywords for pseudocode listings
% \lstdefinelanguage{pseudocodigo}{%
\lstdefinelanguage{pseudocode}{%
    % morekeywords={se,então,para,cada,faça,fim,função,retorna,enquanto,não,is,ou,e,senão,novo,termina,continua,verdadeiro,falso},
    morekeywords={if,then,for,each,do,end,function,return,while,not,is,or,and,else,new,break,continue,true,false},
    sensitive=true, % case-sensitive keywords?
    morecomment=[l]{\#}, % comments are lines that start with a #
}

% display list of figures, tables or listings only if at least one component is present
\usepackage[figure,table,lstlisting]{totalcount}

% hyperlinks
\usepackage{hyperref}
\hypersetup{%
  portuguese,
  pdfencoding=auto,
  breaklinks=true,
  bookmarksopen=true,
  colorlinks=true,
}

\usepackage{makecell}

\makelosymbols{}
\makeloabbreviations{}

% include only the following chapters (faster typesetting)
% \includeonly{chapter06}

\begin{document}
  \title{Análise do Rastro de Proveniência em Simulações Computacionais em Larga Escala}
  \foreigntitle{Provenance Tracing Analysis on Large-Scale Computer Simulations}

  \author{Thiago}{Barroso Perrotta}
  \advisor{Profa.}{Marta}{Lima de Queirós Mattoso}{D.Sc.}
  \advisor{}{Vítor}{Silva Sousa}{M.Sc.}

  % TODO: include examiners in the latest revision
  \examiner{Prof.}{Foo Bar Baz}{D.Sc.}
  \examiner{Prof.}{Alpha Bravo Charlie}{Ph.D.}

  \department{PESC}

  \date{08}{2017}

  \keyword{\textit{Workflows} Científicos}
  \keyword{Análise de Dados Científicos}
  \keyword{Gerência de Fluxos de Dados}
  \keyword{Dados de Proveniência}
  \keyword{Bancos de Dados}

  \foreignkeyword{Scientific Workflows}
  \foreignkeyword{Raw Data Analysis}
  \foreignkeyword{Dataflow Management}
  \foreignkeyword{Provenance Data}
  \foreignkeyword{Databases}

  \listoftodos{}

  \maketitle
  \frontmatter{}

  % !TEX encoding = UTF-8 Unicode
% -*- coding: UTF-8; -*-
% vim: set fenc=utf-8

\dedication{Dedico esse trabalho aos meus pais.}

\chapter*{Agradecimentos}

Agradeço a minha mãe e ao meu pai, por todo o carinho e apoio que recebi, por serem meu exemplo de vida, e por sempre permanecerem presentes ao meu lado, tanto nas fases boas quanto nas ruins.

Agradeço ao Vítor Silva e à professora Marta Mattoso, meus orientadores, por todo o inestimável suporte e atenção nas diversas reuniões de projeto que tivemos, essenciais para o êxito desse trabalho.

Agradeço à professora Márcia Cerioli e ao professor Ricardo Marroquim, pela orientação durante meu período de iniciação científica na UFRJ.

Agradeço ao meu amigo Bruno Buss pelo voto de confiança.

Agradeço ao meu \textit{ex-host} na Google, Andy Venikov, por toda a liderança e motivação.

Também agradeço ao Donald Knuth e ao Leslie Lamport, por tornarem agradável a composição desse trabalho através do \TeX{} e do \LaTeX{}, respectivamente.

Por fim, agradeço aos meus bons amigos, os quais dia a dia fazem-me crescer e tornar-me uma pessoa cada vez melhor, e à serendipidade da vida.

\begin{center}
	\so{Muito obrigado a todos vocês!}
\end{center}

\vfill{}

% https://tex.stackexchange.com/questions/53377/inspirational-quote-at-start-of-chapter
\epigraph{\textit{``Really pay attention to negative feedback and solicit it, particularly from friends\ldots{} hardly anyone does that, and it's incredibly helpful.''}}{--- \textsc{Elon Musk}}


  % !TEX encoding = UTF-8 Unicode
% -*- coding: UTF-8; -*-
% vim: set fenc=utf-8

\begin{abstract}

%REVIEW: resumo em português
Simulações computacionais frequentemente consomem e produzem um grande volume de dados. Parte desses dados é armazenado em arquivos brutos %silva2015analyzing
, que podem assumir diversos formatos, de acordo com o domínio da aplicação, %silva2017raw
tal como o \textit{HDF5} em simulações de dinâmica de fluidos computacionais. %silva2017raw
Durante essas simulações, usuários em geral precisam rastrear grandezas de interesse (como resíduos, tempo de execução e estimativas de erros) %silva2016situ
com base em elementos de dados relacionados de diversos arquivos, a fim de controlar sua execução ao máximo possível %silva2016situ.
No entanto, esse rastreio é realizado usualmente somente ao fim da simulação. %silva2016situ
Apesar dos formatos comuns de arquivo serem apoiados por diversas linguagens de programação e bibliotecas,
os usuários comumente precisam desenvolver programas \textit{ad-hoc} para a análise em grande escala dos mesmos %silva2015analyzing,
que pode ser ainda mais custoso se realizado com bancos de dados, pois eles exigem que os dados \silva[brutos]{ao invés de bruto, utilizar dado científico}
sejam estruturados e carregados em memória. %silva2015analyzing
Nesse cenário, Sistemas de Gerência de Workflows Científicos (\textit{SGWfC}) com ciência do fluxo de dados têm empregado a abstração de workflows científicos
para apoiar a execução paralela de simulações computacionais, em que os dados de proveniência podem favorecer a gerência de elementos de dados
relacionados em múltiplos arquivos de dados científicos (ou seja, produzidos por diferentes programas de simulação) %silva2015analyzing.
Esse trabalho utiliza a arquitetura de componentes ARMFUL, que utiliza uma abordagem baseada em dados de proveniência para extrair e
relacionar grandezas de interesse. Mais especificamente, esse trabalho contribui com um mecanismo de processamento de consultas durante a execução de simulações computacionais,
considerando-se o fluxo de elementos de dados ao longo das simulações computacionais.

\end{abstract}

\begin{foreignabstract}

%REVIEW: resumo em inglês
\silva[Computer simulations potentially receive and produce lots of raw data files.
These files usually follow a \textit{de facto} standard data format established by the application domain,
such as HDF5 in fluid dynamics simulations.
In these simulations, users often need to track quantities of interest (residuals, execution times, error estimates) in order to control as much as possible its execution.
However, this tracking is usually done only once the simulation ends.
This work presents a solution based on provenance data with the purpose of online (\textit{in situ}) extraction and connection among quantities of interest.]{traduzir novamente}

\end{foreignabstract}


  \tableofcontents
  
  \iftotalfigures
  \listoffigures
  \fi
  
  \iftotaltables
  \listoftables
  \fi
  
  \iftotallstlistings
  \renewcommand{\lstlistlistingname}{Lista de Códigos}
  \renewcommand{\lstlistingname}{Código}
  \clearpage
  \addcontentsline{toc}{chapter}{\lstlistlistingname}
  \lstlistoflistings{}
  \fi

  \printloabbreviations{}
  \printlosymbols{}

  \mainmatter{}
  \chapter{Introdução}

O protocolo HTTP (Hypertext Transfer Protocol).
 % Introduction
  % !TEX encoding = UTF-8 Unicode
% -*- coding: UTF-8; -*-
% vim: set fenc=utf-8

% REVIEW: {referencial, embasamento} teórico
\chapter{Referencial Teórico}%
\label{chap:referencial-teorico}

\perrotta{Capítulo 02 --- Referencial Teórico}

Nessa seção serão abordados alguns conceitos e a terminologia que será utilizada ao longo desse trabalho.

\section{Simulação Computacional}

\section{Dataflow}

Uma especificação de fluxo de dados \( D_F \) é definida como \[ D_F = (T, S, \Phi) \] onde:
\begin{itemize}
    \item \( T = \{t_1, t_2, \ldots, t_{\alpha}\} \) \vdots{}
    \( \{t_i \mid i \in \{{1, 2, \ldots, \alpha}\} \} \)
    é transformação
    \item \( S = \{s_1, s_2, \ldots, t_{\beta}\} \) \vdots{}
    \( \{s_i \mid i \in \{{1, 2, \ldots, \beta}\} \} \)
    é conjunto de dados
    \item \( \Phi = \{\phi_1, \phi_2, \ldots, \phi_{\gamma}\} \) \vdots{}
    \( \{\phi_i \mid i \in \{{1, 2, \ldots, \gamma}\} \} \){}
    é dependência de dados
\end{itemize}

\missingfigure{Exemplo de -> o ->, i.e., transformação com dataset de entrada e de saída}

\perrotta{Acho que falta definir o que é uma transformação.}

\subsection{Dependência de dados}

Uma especificação de dependência de dados \( \phi \) é definida como \[ \phi = (s, t_{\textrm{previous}}, t_{\textrm{next}}) \] onde:
\begin{itemize}
    \item \( s \) é um conjunto de dados
    \item \( t_{\textup{previous}} \) é a transformação que {\bf produz} dados para o conjunto de dados \( s \)
    \item \( t_{\textup{next}} \) é a transformação que {\bf consome} dados do conjunto de dados \( s \)
\end{itemize}

\subsection{Conjunto de dados}

A especificação de um conjunto de dados \( s \) é dada por \[ s = (A, C) \] onde:
\begin{itemize}
    \item \( A = \{a_1, a_2, \ldots, a_{\delta} \} \; \vdots{} \; \{ a_i \mid i \in \{1, 2, \ldots, \delta \} \} \) é atributo
    \item \( C = \{c_1, c_2, \ldots, c_{\zeta} \} \; \vdots{} \; \{ c_i \mid i \in \{1, 2, \ldots, \zeta \} \} \) é coleção de dados
\end{itemize}

\subsection{Atributo}

Um atributo é especificado da seguinte forma: \[ a = (\textrm{nome},\textrm{tipo}) \therefore \textrm{tipo} = \{\textup{inteiro}, \textup{ponto flutuante}, \textup{texto}, \textup{arquivo}\} \]

\subsection{Coleção de dados}

Uma coleção de dados é definida como \[ c = \{ i_1, i_2, \ldots, i_{\eta} \} \; \vdots \; \{ i_j \mid j \in \{ 1, 2, \ldots, \eta \} \} \] é item de dados

\subsection{Item de dados}

A especificação de um item de dados é dada por \[ i = (s, T^{\ast}_{\textrm{previous}}, T^{\ast}_{\textrm{next}}, e) \] onde:

\begin{itemize}
    \item \( s \) é um conjunto de dados
    \item \( T^{\ast}_{\textrm{previous}} \) é o conjunto com todas as instâncias de todas as transformações que são responsáveis por gerar \( i \)
    \item \( T^{\ast}_{\textrm{next}} \) é o conjunto com todas as instâncias de todas as transformações que consomem \( i \)
    \item \( e \) é um elemento de dados
\end{itemize}

\subsection{Elemento de dados}

Um elemento de dados é definido como

\[ e = \{ v_1, v_2, \ldots, v_{\theta} \} \therefore \#(s.A) = \#(e) \]
\[ v_{\theta} \; \textrm{é o valor do atributo} \; a_{\theta} \; \textrm{de um item de dados do conjunto de dados} \; s \]

onde \( \#(S) \) representa a cardinalidade do conjunto \( S \), i.e.\ a quantidade de elementos presentes no mesmo.

\subsection{Um exemplo}

Para ilustrar as definições e especificações da \autoref{sec:especificacoes}, vamos utilizar uma pequena amostra do conjunto de dados de sedimentação. Na \autoref{fig:example-dataflow-specification} \perrotta{expandir comentário sobre a figura}

\begin{figure}[ht]
    \centering
    \includegraphics[width=0.35\textwidth]{img/example-dataflow-specification}
    \caption[Exemplo de especificação de fluxo de dados]{Exemplo de especificação de fluxo de dados, com duas transformações e três conjuntos de dados.}%
    \label{fig:example-dataflow-specification}
\end{figure}

\missingfigure{Incluir um exemplo, com figuras, de um pequeno dataflow.}

\[
\begin{aligned}
D_F = && (T, S, \Phi) \\
T = && \{ \textrm{inputmesh} \} \\
S = && \{ \textrm{iinputmesh}, \textrm{oinputmesh} \} \\
\Phi = && \{ (\textrm{nil}, \textrm{iinputmesh}, \textrm{inputmesh}),
(\textrm{inputmesh}, \textrm{oinputmesh}, \textrm{nil})
\} \\
\end{aligned}
\]

\perrotta{Capítulo 02 --- misc}

%%%%%%%%%%%%%%%%%%%%%%
% Simulações computacionais
% (OK) Fluxo de dados = dataflow - definições {d,n}o outro google docs
\section{Proveniência de dados}
\subsection{Categorias de dados de proveniência}

\section{Gerência do fluxo de dados}
\subsection{Físico}
\subsection{Lógico}
% Gerência do fluxo de dados
% Mecanismos para processamento de consultas

  % !TEX encoding = UTF-8 Unicode
% -*- coding: UTF-8; -*-
% vim: set fenc=utf-8

\chapter{Trabalhos Relacionados}%
\label{chap:trabalhos-relacionados}

Dado que o principal objetivo dessa monografia consiste no desenvolvimento de um processador de consultas integrado à arquitetura ARMFUL, que será abordada no \autoref{chap:arquitetura-armful}, esse capítulo aborda alguns trabalhos e publicações relacionados a ela, no que diz respeito às simulações computacionais, aos \textit{workflows} científicos, aos dados de proveniência e, em especial, aos tipos de consulta que podem ser realizados.

\section{Tipos de consulta}%
\label{sec:tipos-de-consulta}

Conforme introduzimos na \autoref{sec:motivacao}, existem três tipos básicos de consultas que podem ser realizados em uma base de dados de proveniência de uma simulação computacional no cenário de análise exploratória de dados científicos~\cite{silva2015analyzing,silva2015propostadoutorado}:

\begin{enumerate}
    \item análise de dados científicos isolados de um único arquivo, envolvendo a extração de conteúdos específicos de domínio, que é abordada na \autoref{sec:analise-de-dados-cientificos-isolados};
    \item rastreamento de fluxos de múltiplos arquivos relacionados através de transformações de dados correspondentes, discutido na \autoref{sec:rastreamento-de-fluxos-de-arquivos}; e
    \item rastreamento de elementos de dados relacionados em múltiplos arquivos, que será elucidado na \autoref{sec:rastreamento-de-elemento-de-dados-em-multiplos-arquivos}.
\end{enumerate}

As consultas do tipo 1 requerem apenas acesso a um único arquivo, enquanto consultas dos tipos 2 e 3 necessitam da gerência do fluxo de dados da simulação computacional~\cite{silva2015analyzing}.

% \subsection{Um exemplo}

% A \autoref{fig:types-of-queries-1-2-3} apresenta um exemplo que envolve os três tipos de consulta citados na \autoref{sec:tipos-de-consulta}:

% \begin{itemize}
%     \item exemplo de consulta do tipo 1: usuário realiza uma consulta específica do domínio ao conteúdo do arquivo \mbox{\texttt{projected{\_}images.tbl}}. Os atributos FA e FB (marcados com linhas vermelhas) são acessados e capturados para a subsequente análise de transformações lineares realizadas pelo programa de projeção presente na simulação computacional com uma configuração pré-definida;
%     \item exemplo de consulta do tipo 2: as setas de cor preta representam a sequência do fluxo de arquivos nos formatos \texttt{TBL} e \texttt{FITS}, os quais são projetados em outros arquivos \texttt{TBL} e depois transformados para eventualmente criar um arquivo de imagem no formato JPG (\mbox{\texttt{mosaic.jpg}}). O registro desses relacionamentos permite a possibilidade do rastreamento dos arquivos intermediários que geram o arquivo de imagem.
%     \item exemplo de consulta do tipo 3: podemos visualizar um dos fluxos de elementos de dados através da representação das setas na cor vermelha. Nesse exemplo, o atributo CRVAL1 é utilizado como chave para relacionar os dados científicos em diferentes arquivos. Dessa forma, através desse atributo, o arquivo \mbox{\texttt{hdu{\_}1n.fits}} pode ser relacionado ao elemento de dados FB cujo valor é 0.001969140 do arquivo \mbox{\texttt{projected{\_}images.tbl}}.
% \end{itemize}

% \begin{figure}[htb]
%     \centering
%     \includegraphics[width=\textwidth]{img/types-of-queries-1-2-3}
%     \caption[Exemplo de análise de uma simulação de astronomia]{Exemplo de uma simulação de astronomia para análise de arquivos de dados científicos. Encontrada originalmente em~\cite{silva2015propostadoutorado}.}%
%     \label{fig:types-of-queries-1-2-3}
% \end{figure}

\section{Análise de dados científicos isolados}%
\label{sec:analise-de-dados-cientificos-isolados}

Consultas do \textbf{tipo 1} são caracterizadas pela \textbf{análise de dados científicos}, encontrados em arquivos que foram produzidos por um programa de simulação, e envolve tipicamente a \textbf{extração} (do inglês: \textit{parsing}) ou \textbf{interpretação} desses dados de formatos de arquivos que seguem um padrão de acordo com domínio da simulação computacional. Nesse sentido, é necessário conhecer a estrutura e~/~ou a codificação utilizada nesses arquivos para que seja possível acessá-los e extrair um significado semântico dos dados do mesmo. Além disso, cada formato de arquivo possui programas específicos e peculiares para a sua análise e extração. Por exemplo, um arquivo em formato de imagem tal como o \abbrev{PNG}{\textit{Portable Network Graphics}} PNG (\textit{Portable Network Graphics}) pode ser analisado (visualizado) por um editor de imagens, enquanto que um arquivo em formato de música tal como o \abbrev{WAV}{Waveform Audio File Format} WAV (\textit{Waveform Audio File Format}) pode ser analisado (reproduzido) por um \textit{player} de áudio. Em particular, destaca-se que a semântica da análise --- reprodução, visualização, extração, etc. --- é inerente ao tipo de arquivo analisado.

Esses dados e seus respectivos arquivos podem estar ou não em formato binário; por exemplo, um arquivo em formato \abbrev{JSON}{JavaScript Object Notation} JSON (\textit{JavaScript Object Notation}) geralmente contém dados em formato de texto; enquanto que um arquivo FITS (\textit{Flexible Image Transport System})~\cite{greisen2002representations}, utilizado em astronomia, pode possuir parte de seus dados em formato binário, assim como o NetCDF (\textit{Network Common Data Form})~\cite{rew1990netcdf}, utilizado em dinâmica de fluidos computacionais, e o HDF5~\cite{folk1999hdf5}, utilizado em várias aplicações distintas.

Para apoiar a extração de dados científicos, alguns trabalhos têm utilizado técnicas de indexação de dados científicos no conteúdo de arquivos e carregado tais dados em repositórios externos (arquivos com índices e dados no formato binário para proporcionar um acesso eficiente) ou uma base de dados, como o NoDB~\cite{alagiannis2012nodb}, RAW~\cite{karpathiotakis2014adaptive}, FastBit~\cite{wu2009fastbit}, FastQuery~\cite{chou2011parallel} e SDS/Q~\cite{blanas2014parallel}. Entretanto, todos esses trabalhos relacionados apresentam um processador de consultas limitado ao conteúdo de arquivos de forma isolada. Ou seja, não é possível relacionar dados de múltiplos arquivos de dados científicos.

Logo, as consultas do tipo 1 são as mais simples de serem realizadas em relação às outras, pois elas são auto-contidas no que diz respeito aos dados científicos, isto é, apenas a presença desses dados é suficiente para que esse tipo de consulta possa ser realizado: não é necessária a especificação sobre o fluxo de dados da simulação computacional.

\section{Análise do fluxo de arquivos}%
\label{sec:rastreamento-de-fluxos-de-arquivos}

Consultas do \textbf{tipo 2} são caracterizadas pelo rastro do \textbf{fluxo de arquivos}, que se relacionam através de transformações de dados. Elas são frequentemente apoiadas por sistemas de \textit{workflows} científicos, tais como o Chiron~\cite{ogasawara2011algebraic}, o Pegasus~\cite{deelman2005pegasus} e o Kepler~\cite{ludascher2006scientific}. Através desse tipo de consulta, que busca informações de proveniência retrospectiva, os usuários são capazes de rastrear, em uma míriade de arquivos de uma simulação computacional em larga escala, qual foi o fluxo de arquivos que gerou determinados conjuntos de dados finais a partir de quais conjuntos de dados científicos iniciais.

Um exemplo de ferramenta que é capaz de realizar consultas do tipo 2 é o NoWorkflow~\cite{murta2014noworkflow} (do inglês: \textit{\textbf{n}ot \textbf{o}nly \textbf{workflow}}), que captura de forma transparente a proveniência de \textit{scripts} escritos na linguagem de programação Python. Essa captura é feita a nível de funções (subrotinas) da linguagem e, em particular, chamadas de sistema do tipo \emph{open} (=leitura de arquivos) presentes nos \textit{scripts} são capturadas, assim como os arquivos que lhe são passados como argumentos, que então são armazenados em um banco de dados relacional que identifica as relações (de entrada e de saída) entre esses arquivos. Dessa maneira, o NoWorkflow possui toda a informação de proveniência retrospectiva~\cite{Pimentel2016} necessária para consultar o fluxo dos arquivos lidos e escritos durante a execução dos \textit{scripts}\footnote{Nesse contexto, um \textit{script} é análogo a um conjunto de transformações de dados, representando um programa de simulação.}.

Outra ferramenta que suporta consultas do tipo 2 é o YesWorkflow~\cite{mcphillips2015yesworkflow}, que permite que os usuários façam anotações na forma de comentários especiais em seus \textit{scripts} e subsequentemente extrai e analisa esses comentários, representando os \textit{scripts} na forma de entidades e relações entre elas, formando e definindo um fluxo de dados. O YesWorkflow pode ser utilizado em qualquer \textit{script}, não se limitando somente a Python, diferentemente do NoWorkflow; mas, em contrapartida, ele requer anotações explícitas da parte do usuário, não sendo capaz de extrair informações de proveniência automaticamente como o NoWorkflow~\cite{Pimentel2016}. Os comentários etiquetam regiões arbitrárias dos \textit{scripts}, associando valores a elas, os quais podem correlacionar e fazer associações a outras regiões; desse modo, um conjunto de comentários é capaz de especificar o fluxo de arquivos do fluxo de dados.

Citamos mais um exemplo de ferramenta que é capaz de realizar consultas do tipo 2, o Tigres~\cite{hendrix2016tigres}, que é inspirado no paradigma \textit{MapReduce} e suporta a criação de fluxos de dados sequenciais e paralelos. O Tigres é utilizado através de uma \abbrev{API}{Application Programming Interface} API (do inglês, \textit{Application Programming Interface}) de \textit{templates} escrita em Python, com a qual o usuário define elementos do fluxo de dados e monitora a execução de programas científicos. Esses \textit{templates} funcionam como blocos de construção, que executam tarefas sequencialmente e geram informações de dependência e de proveniência entre os mesmos, tanto a nível prospectivo quanto a nível retrospectivo.

\section{Análise do fluxo de elementos de dados em múltiplos arquivos}%
\label{sec:rastreamento-de-elemento-de-dados-em-multiplos-arquivos}

Consultas do \textbf{tipo 3} envolvem o rastreamento de \textbf{elementos de dados relacionados} em \textbf{múltiplos arquivos}. Nesse sentido, são uma extensão das consultas do tipo 2, uma vez que é necessária uma granularidade mais fina para gerenciar o fluxo de dados, no nível de elementos de dados, e não apenas no nível do fluxo de arquivos.

O estado da arte no que diz respeito às soluções para a análise de dados científicos em simulações científicas concentra-se hoje no apoio a consultas do tipo 1~\cite{alagiannis2012nodb,karpathiotakis2014adaptive,wu2009fastbit,folk1999hdf5,silva2015propostadoutorado} e do tipo 2~\cite{murta2014noworkflow,mcphillips2015yesworkflow,hendrix2016tigres,Pimentel2016}, como vimos em alguns exemplos nas seções anteriores. No entanto, algumas soluções para consultas do tipo 3 foram apresentadas nos últimos anos e vêm se consolidando recentemente.

Por exemplo, o NoWorkflow, mencionado na \autoref{sec:rastreamento-de-fluxos-de-arquivos}, também possui suporte a consultas do tipo 3, permitindo o rastreamento de elementos de dados em nível de variáveis, de funções e de classes em \textit{scripts}, porém ele se limita somente a \textit{scripts} escritos na linguagem Python~\cite{murta2014noworkflow}. Além dele, a biblioteca Tigres também permite a gerência dos elementos de dados ao relacionar os valores de atributos consumidos e produzidos por cada tarefa computacional.

Outro paradigma que suporta consultas do tipo 3 é o da arquitetura ARMFUL, que é baseada em componentes e é instanciada, por exemplo, como o \textit{DfAnalyzer} em~\cite{silva2017raw}. Como esta monografia concentra-se no desenvolvimento de um processador de consultas no \textit{DfAnalyzer}, essa ferramenta é abordada detalhadamente no \autoref{chap:arquitetura-armful}.
  % !TEX encoding = UTF-8 Unicode
% -*- coding: UTF-8; -*-
% vim: set fenc=utf-8

\chapter{Arquitetura ARMFUL}%
\label{chap:arquitetura-armful}

\perrotta{Capítulo 04 --- Arquitetura ARMFUL}

  % !TEX encoding = UTF-8 Unicode
% -*- coding: UTF-8; -*-
% vim: set fenc=utf-8

\chapter{Abordagem para análise de rastros de proveniência}

\perrotta{Capítulo 05 - Abordagem para análise de rastros de proveniência}

  % !TEX encoding = UTF-8 Unicode
% -*- coding: UTF-8; -*-
% vim: set fenc=utf-8

\chapter{Experimentos}%
\label{chap:experimentos}

Esse capítulo apresenta alguns exemplos de experimentos e de simulações computacionais com base em uma aplicação do DfAnalyzer no campo de sedimentação de dinâmica de fluidos computacionais~\cite{silva2016situ}, os quais foram executados a fim de ilustrar a implementação e a utilização do Query Processor --- apresentado no \autoref{chap:rastros-de-proveniencia} ---, assim como os resultados obtidos com as consultas realizadas nos mesmos.

\section{Simulação computacional em sedimentação}

O DfAnalyzer (\textit{c.f.} \autoref{sec:dfanalyzer-uma-instancia-da-arquitetura-armful}) utiliza a \textit{libMesh}, uma biblioteca \textit{open-source} implementada na linguagem C++, desenvolvida para facilitar a simulação de aplicações paralelas de refinamento de malhas e elementos finitos adaptativos~\cite{boncz2008breaking}, em um resolvedor chamado \textit{libMesh-sedimentation}. O propósito dessa aplicação é simular a turbidez e perturbação de correntes de fluidos tipicamente encontradas em processos geológicos. Os sedimentos que são transportados devido à dinâmica e ao movimento dos fluidos computacionais são descritos por um modelo matemático que resulta da equação de incompressibilidade de Navier-Stokes (fluido) combinada com uma equação de transporte dominada por advecção (concentração de sedimentos). A \textit{libMesh-sedimentation} emprega um método de elementos finitos de multi-escala variacional no qual uma abordagem escalonada é utilizada para representar e simular a evolução do tempo nas equações de acoplamento entre o fluido e os sedimentos. 

Nessa aplicação os usuários empregam simulações complexas nas quais grandezas de interesse, tais como resíduos e estimativas de erros, são utilizadas para controlar de forma fina o êxito e a performance da execução~\cite{silva2016situ}. Por exemplo, a convergência ou divergência de valores de determinadas grandezas e atributos, ao longo do tempo, é uma potencial e rica fonte de informação sobre o andamento da simulação computacional. Contudo, em geral, não basta analisar uma única grandeza: frequentemente faz-se necessária a análise dos dados científicos de múltiplos arquivos, gerados em diferentes passos durante a execução do \textit{solver}. Nesse cenário a análise é possibilitada pelo DfAnalyzer o qual, devido a sua arquitetura em componentes herdada da ARMFUL, permite que consultas relacionadas a proveniência e aos dados científicos de múltiplos arquivos e transformações de dados sejam realizadas \textit{on-line}, ainda durante a execução da simulação computacional.

Neste capítulo realizamos várias consultas com um fluxo de dados \(D^{\dagger}\) baseado no \textit{solver} de \textbf{simulação de sedimentação de dinâmica de fluidos computacionais} mencionado nos parágrafos anteriores. Esse fluxo de dados \(D^{\dagger}\) está ilustrado na \autoref{fig:experiments-dataflow}.

\begin{figure}[!htb]
    \centering
    \includegraphics[width=\textwidth]{img/experiments-dataflow}
    \caption[Fluxo de dados $D^{\dagger}$ utilizado nos experimentos]{Fluxo de dados $D^{\dagger}$ utilizado nos experimentos e nas consultas do \autoref{chap:experimentos}.}%
    \label{fig:experiments-dataflow}
\end{figure}

Devido à grande quantidade de conjuntos de dados \(S^{\dagger}\) de \(D^{\dagger}\), não listamos todos os atributos de dados de \(S^{\dagger}\). No entanto, para fins de ilustração, alguns atributos de dados de \(S^{\dagger}\) podem ser visualizados na \autoref{tab:experiments-data-attributes}.

\begin{table}[htb]
    \centering
    \begin{tabular}{c|c|c|c}
\textbf{Conjunto de dados}                  & \textbf{Atributo de dados} & \textbf{Tipo}   & \makecell{\textbf{Exemplo} \\ \textbf{de Valor}}             \\ \hline
\multirow{3}{*}{osolversimulationtransport} & time                       & \makecell{ponto \\ flutuante} & 1e-05                                  \\ \cline{2-4}
                                            & t\_step                    & inteiro         & 0                                      \\ \cline{2-4}
                                            & meshwriter\_task\_id       & inteiro         & 17                                     \\ \hline
\multirow{2}{*}{omeshaggregator}            & xdmf                       & arquivo         & \texttt{\textasciitilde/output\_48.xmf}             \\ \cline{2-4}
                                            & n\_processors              & inteiro         & 480                                    \\ \hline
omeshrefinement                             & first\_step\_refinement    & booleano        & falso                                  \\ \hline
ovisualization                              & png                        & arquivo         & \texttt{\textasciitilde/image\_99.png} \\ \hline
oinputmesh                                  & mesh\_file                 & arquivo         & \texttt{\textasciitilde/necker3d.mesh}               \\ \hline
otimestepcontrolconfig                      & model\_name                & string          & PC11                                  
    \end{tabular}
    \caption[Exemplos de alguns atributos de dados de \(S^{\dagger}\)]{Exemplos de alguns atributos de dados de \(S^{\dagger}\).}%
    \label{tab:experiments-data-attributes}
\end{table}

% \silva{REVIEW: que tipos de arquivos ou dados são coletados na libMesh-sedimentation?. No solver, capturamos dados referentes ao resíduo do solver ao resolver as equações para o fluido e os sedimentos. Para a gerência de arquivos, consideramos os raw data files nos formatos XDMF e HDF5.}

\perrotta{TODO: citar artigo do email}

\section{Experimentos}

As consultas dos experimentos das próximas subseções, relacionadas ao Query Processor, foram executadas em um computador com as seguintes especificações:

\begin{itemize}
	\item Macbook Pro Retina 2015, com o sistema operacional macOS Sierra 10.12.6;
    \item Processador Intel Core i5 2,7~GHz, com 4~CPUs;
    \item 8~GB de memória RAM do tipo DDR3.
\end{itemize}

Quatro consultas foram realizadas, cada uma das quais com diversos parâmetros para a função \texttt{generateSqlQuery} (\textit{c.f.} \autoref{subsec:geracao-da-consulta-em-sql}).

\subsection{Consulta \#1}

% Query gerada:
%
% SELECT osolversimulationtransport.time, A% VG(oline2extraction.d)
% FROM osolversimulationtransport, oline2extraction, omeshwriter
% WHERE (osolversimulationtransport.line2extraction_task_id = oline2extraction.line2extraction_task_id)
% AND (osolversimulationtransport.meshwriter_task_id = omeshwriter.meshwriter_task_id);
% GROUP BY osolversimulationtransport.time;

% Query gerada (com GROUP BY):
%
% SELECT osolversimulationtransport.time, AVG(oline2extraction.d)
% FROM osolversimulationtransport, oline2extraction, omeshwriter
% WHERE (osolversimulationtransport.line2extraction_task_id = oline2extraction.line2extraction_task_id)
% AND (osolversimulationtransport.meshwriter_task_id = omeshwriter.meshwriter_task_id)
% GROUP BY osolversimulationtransport.time;

O objetivo da primeira consulta é a análise da média da concentração de sedimentos em uma linha extraída (\textit{i.e.}, um conjunto de pontos) de arquivos científicos, variando o tempo da simulação. Ela envolve a inspeção de três conjuntos de dados (\texttt{osolversimulationtransport}, \texttt{oline2extraction} e \texttt{omeshwriter}) de \(S^{\dagger}\) e emprega o mapeamento de atributos de dados para rastro de proveniência do tipo físico (\textit{c.f.} \autoref{subsec:rastro-de-proveniencia-do-tipo-fisico}). Apenas duas projeções são de interesse (\texttt{osolversimulation.time} e a média aritmética de \texttt{oline2extraction.d}). A especificação completa da consulta está disponível na \autoref{tab:experiments-1-especificacao}.

\begin{table}[htb]
    \centering
    \begin{tabular}{c|c}
\textbf{Argumento}          & \textbf{Valor} \\ \hline
\texttt{D}                  & $D^{\dagger}$ \\
\texttt{dsOrigins}          & \{\texttt{osolversimulationtransport}\} \\
\texttt{dsDestinations}     & \{\texttt{oline2extraction}, \texttt{omeshwriter}\} \\
\texttt{type}               & physical \\
\texttt{projections}        & \{\texttt{osolversimulationtransport.time}, \texttt{AVG(oline2extraction.d)}\} \\
\texttt{selections}         & \varnothing \\
\texttt{dsIncludes}         & \varnothing \\
\texttt{dsExcludes}         & \varnothing \\
    \end{tabular}
    \caption[Argumentos da função \texttt{generateSqlQuery} para a consulta \#1]{Especificação dos argumentos da função \texttt{generateSqlQuery} para a consulta~\#1.}%
    \label{tab:experiments-1-especificacao}
\end{table}

A consulta \#1 em SQL gerada pelo Query Processor está listada no \autoref{lst:experiments-1-sql}. \textbf{Observação:} uma vez que a função de média aritmética (\texttt{AVG}) é utilizada na projeção, faz-se necessária a utilização da cláusula \texttt{GROUP BY} da linguagem SQL para que a consulta \#1 fique sintaticamente válida e bem definida. Contudo, como o Query Processor não suporta essa cláusula, ela foi adicionada manualmente à consulta gerada pela função \texttt{generateSqlQuery}.

\begin{minipage}[c]{0.95\textwidth}
\begin{lstlisting}[language=sql,label={lst:experiments-1-sql},caption={[Código em SQL gerado na consulta~\#1]Código em SQL gerado na consulta~\#1 (tempo médio: 40,29~ms).}]
SELECT osolversimulationtransport.time, AVG(oline2extraction.d)
FROM osolversimulationtransport, oline2extraction, omeshwriter
WHERE (osolversimulationtransport.line2extraction_task_id = oline2extraction.line2extraction_task_id) 
AND (osolversimulationtransport.meshwriter_task_id = omeshwriter.meshwriter_task_id)
GROUP BY osolversimulationtransport.time;
\end{lstlisting}
\end{minipage}

O resultado da consulta \#1, executada na base de dados de \(D^{\dagger}\) carregada no MonetDB, pode ser visualizado no \autoref{lst:experiments-1-sqlresults}. Uma representação gráfica dos conjuntos de dados de \(S^{\dagger}\) utilizados nessa consulta estão realçados em vermelho na \autoref{fig:experiments-dataflow-1}.

\begin{lstlisting}[language=sqlresults,label={lst:experiments-1-sqlresults},caption={[Resultados da consulta \#1.]Resultados da consulta \#1 (7 tuplas, tempo médio: 4,346~ms).}]
+--------------------------+--------------------------+
| time                     | L71                      |
+==========================+==========================+
|                1.3398483 |     1.74129306930693e-44 |
|                3.1009347 |  -1.0847045643564339e-44 |
|                5.4124618 |    7.853749801980205e-39 |
|                7.8695609 |  -1.8180726336633645e-33 |
|               10.1307669 |   1.0729463267326738e-27 |
|               12.6055519 |   1.0473414950495052e-24 |
|                       15 |   -8.768540693069305e-22 |
+--------------------------+--------------------------+
\end{lstlisting}

\begin{figure}[htb]
    \centering
    \includegraphics[width=\textwidth]{img/experiments-dataflow-1}
    \caption[Caminho do fluxo de dados \(D^{\dagger}\) rastreado na consulta \#1]{Caminho do fluxo de dados \(D^{\dagger}\) rastreado na consulta \#1.}%
    \label{fig:experiments-dataflow-1}
\end{figure}

\subsection{Consulta \#2}

% Query gerada:
%
% SELECT osolversimulationtransport.time, oline0extraction.points0, oline0extraction.points1, oline0extraction.points2, oline0extraction.d
% FROM osolversimulationtransport, oline0extraction, omeshwriter
% WHERE (osolversimulationtransport.time < 5.5)
% AND (oline0extraction.d > 0.1)
% AND (osolversimulationtransport.line0extraction_task_id = oline0extraction.line0extraction_task_id)
% AND (osolversimulationtransport.meshwriter_task_id = omeshwriter.meshwriter_task_id);

% Query gerada (com LIMIT):
%
% SELECT osolversimulationtransport.time, oline0extraction.points0, oline0extraction.points1, oline0extraction.points2, oline0extraction.d
% FROM osolversimulationtransport, oline0extraction, omeshwriter
% WHERE (osolversimulationtransport.time < 5.5)
% AND (oline0extraction.d > 0.1)
% AND (osolversimulationtransport.line0extraction_task_id = oline0extraction.line0extraction_task_id)
% AND (osolversimulationtransport.meshwriter_task_id = omeshwriter.meshwriter_task_id)
% LIMIT 10;

O objetivo da segunda consulta é a análise da concentração de sedimentos em uma linha extraída de arquivos científicos, considerando um instante de tempo fixo e um intervalo específico de valores para a concentração de sedimentos especificada pelo usuário. Ela ilustra a inspeção de três conjuntos de dados (\texttt{osolversimulationtransport}, \texttt{oline0extraction} e \texttt{omeshwriter}) de \(S^{\dagger}\), empregando o rastro de proveniência do tipo físico. Várias projeções são de interesse para o objetivo de analisar o comportamento do conjunto de dados \texttt{oline0extraction} ao longo do tempo (segundo o atributo de dados \texttt{osolversimulationtransporte.time}). A especificação completa da consulta está disponível na \autoref{tab:experiments-2-especificacao}.

\begin{table}[htb]
    \centering
    \begin{tabular}{c|c}
\textbf{Argumento}          & \textbf{Valor} \\ \hline
\texttt{D}                  & $D^{\dagger}$ \\ \hline
\texttt{dsOrigins}          & \{\texttt{osolversimulationtransport}\} \\ \hline
\texttt{dsDestinations}     & \{\texttt{oline0extraction}, \texttt{omeshwriter}\} \\ \hline
\texttt{type}               & physical \\ \hline
\texttt{projections}        & \makecell{\{\texttt{osolversimulationtransport.time}, \\
                                          \texttt{oline0extraction.points0}, \\ 
                                          \texttt{oline0extraction.points1}, \texttt{oline0extraction.points2}, \\
                                          \texttt{oline0extraction.d}\}} \\ \hline
\texttt{selections}         & \makecell{\{\texttt{osolversimulationtransport.time < 5.5}, \\
                                          \texttt{oline0extraction.d > 0.1}\}} \\ \hline
\texttt{dsIncludes}         & \varnothing \\ \hline
\texttt{dsExcludes}         & \varnothing \\
    \end{tabular}
    \caption[Argumentos da função \texttt{generateSqlQuery} para a consulta \#2]{Especificação dos argumentos da função \texttt{generateSqlQuery} para a consulta~\#2.}%
    \label{tab:experiments-2-especificacao}
\end{table}

A consulta \#2 em SQL gerada pelo Query Processor está listada no \autoref{lst:experiments-2-sql}. \textbf{Observação:} como a cláusula \texttt{LIMIT} não é implementada pelo QP, ela foi adicionada manualmente à consulta após sua geração pela função \texttt{generateSqlQuery}.

\begin{minipage}[c]{0.95\textwidth}
\begin{lstlisting}[language=sql,label={lst:experiments-2-sql},caption={[Código em SQL gerado na consulta~\#2]Código em SQL gerado na consulta~\#2 (tempo médio: 15,45~ms).}]
SELECT osolversimulationtransport.time, oline0extraction.points0, oline0extraction.points1, oline0extraction.points2, oline0extraction.d
FROM osolversimulationtransport, oline0extraction, omeshwriter
WHERE (osolversimulationtransport.time < 5.5) 
AND (oline0extraction.d > 0.1) 
AND (osolversimulationtransport.line0extraction_task_id = oline0extraction.line0extraction_task_id) 
AND (osolversimulationtransport.meshwriter_task_id = omeshwriter.meshwriter_task_id)
LIMIT 10;
\end{lstlisting}
\end{minipage}

O resultado da consulta \#2, executada na base de dados de \(D^{\dagger}\) carregada no MonetDB, pode ser visualizado no \autoref{lst:experiments-2-sqlresults}. Uma representação gráfica dos conjuntos de dados de \(S^{\dagger}\) utilizados nessa consulta estão realçados em vermelho na \autoref{fig:experiments-dataflow-2}.

\begin{lstlisting}[language=sqlresults,label={lst:experiments-2-sqlresults},caption={[Resultados da consulta \#2.]Resultados da consulta \#2 (10 tuplas, tempo médio: 5,92~ms).}]
+--------------+-----------+---------+---------+----------+
| time         | points0   | points1 | points2 | d        |
+==============+===========+=========+=========+==========+
|    5.4124618 |         0 |       1 |       0 |  0.10814 |
|    5.4124618 |      0.18 |       1 |       0 |   0.1082 |
|    5.4124618 |      0.36 |       1 |       0 |  0.10825 |
|    5.4124618 |      0.54 |       1 |       0 |  0.10825 |
|    5.4124618 |      0.72 |       1 |       0 |  0.10825 |
|    5.4124618 |         0 |       1 |       0 |  0.10814 |
|    5.4124618 |      0.18 |       1 |       0 |   0.1082 |
|    5.4124618 |      0.36 |       1 |       0 |  0.10825 |
|    5.4124618 |      0.54 |       1 |       0 |  0.10825 |
|    5.4124618 |      0.72 |       1 |       0 |  0.10825 |
+--------------+-----------+---------+---------+----------+
\end{lstlisting}
% |    5.4124618 |         0 |       1 |       0 |  0.10814 |
% |    5.4124618 |      0.18 |       1 |       0 |   0.1082 |
% |    5.4124618 |      0.36 |       1 |       0 |  0.10825 |
% |    5.4124618 |      0.54 |       1 |       0 |  0.10825 |
% |    5.4124618 |      0.72 |       1 |       0 |  0.10825 |
% |    5.4124618 |         0 |       1 |       0 |  0.10814 |
% |    5.4124618 |      0.18 |       1 |       0 |   0.1082 |
% |    5.4124618 |      0.36 |       1 |       0 |  0.10825 |
% |    5.4124618 |      0.54 |       1 |       0 |  0.10825 |
% |    5.4124618 |      0.72 |       1 |       0 |  0.10825 |

\begin{figure}[htb]
    \centering
    \includegraphics[width=\textwidth]{img/experiments-dataflow-2}
    \caption[Caminho do fluxo de dados \(D^{\dagger}\) rastreado na consulta \#2]{Caminho do fluxo de dados \(D^{\dagger}\) rastreado na consulta \#2.}%
    \label{fig:experiments-dataflow-2}
\end{figure}

\subsection{Consulta \#3}

% Query a ser gerada:
%
% SELECT minSeds.time, minSeds.transport_n_linear_iterations,
%     minSeds.transport_final_linear_residual as minResiduals, maxSeds.transport_final_linear_residual as maxResiduals,
%     (maxSeds.transport_final_linear_residual - minSeds.transport_final_linear_residual) as variation
% FROM osolversimulationtransport as minSeds,
%     osolversimulationtransport as maxSeds,
%     (
%     SELECT seds.solversimulationtransport_task_id, seds.time, seds.transport_n_linear_iterations,
%     min(seds.transport_l) as minIter, max(seds.transport_l) as maxIter
%     FROM osolversimulationtransport as seds
%     GROUP BY seds.solversimulationtransport_task_id, seds.time, seds.transport_n_linear_iterations
%     ) as seds
% WHERE seds.time = minSeds.time
% AND seds.time = maxSeds.time
% AND minSeds.transport_l = seds.minIter
% AND maxSeds.transport_l = seds.maxIter
% AND maxSeds.transport_final_linear_residual > minSeds.transport_final_linear_residual;

O objetivo da terceira consulta é a análise da primeira variação negativa dos resíduos lineares finais durante a execução da simulação entre os primeiros e os últimos passos não-lineares. 

% Ela ilustra a inspeção de três conjuntos de dados (\texttt{osolversimulationtransport}, \texttt{oline0extraction} e \texttt{omeshwriter}) de \(S^{\dagger}\), empregando o rastro de proveniência do tipo físico. Várias projeções são de interesse para o objetivo de analisar o comportamento do conjunto de dados \texttt{oline0extraction} ao longo do tempo (segundo o atributo de dados \texttt{osolversimulationtransporte.time}). A especificação completa da consulta está disponível na \autoref{tab:experiments-2-especificacao}.

\begin{table}[htb]
    \centering
    \begin{tabular}{c|c}
\textbf{Argumento}          & \textbf{Valor} \\ \hline
\texttt{D}                  & $D^{\dagger}$ \\ \hline
\texttt{dsOrigins}          & \{\texttt{}\} \\ \hline
\texttt{dsDestinations}     & \{\texttt{}, \texttt{}\} \\ \hline
\texttt{type}               & logical \\ \hline
\texttt{projections}        & \{\texttt{}\} \\ \hline
\texttt{selections}         & \makecell{
                                        \texttt{seds.time = minSeds.time}, \\
                                        \texttt{seds.time = maxSeds.time}, \\
                                        \texttt{minSeds.transport\_l = seds.minIter}, \\
                                        \texttt{maxSeds.transport\_l = seds.maxIter}, \\
                                        \texttt{(maxSeds.transport\_final\_linear\_residual >} \\
                                        \texttt{minSeds.transport\_final\_linear\_residual)}} \\ \hline
\texttt{dsIncludes}         & \varnothing \\ \hline
\texttt{dsExcludes}         & \varnothing \\
    \end{tabular}
    \caption[Argumentos da função \texttt{generateSqlQuery} para a consulta \#3]{Especificação dos argumentos da função \texttt{generateSqlQuery} para a consulta~\#3.}%
    \label{tab:experiments-3-especificacao}
\end{table}

\perrotta{TODO: consulta \#3}

\subsection{Consulta \#4}

\perrotta{TODO: consulta \#4}

% TODO: mention: Tempo: rodar 4 vezes, tirar a média das 3 últimas
  % !TEX encoding = UTF-8 Unicode
% -*- coding: UTF-8; -*-
% vim: set fenc=utf-8

\chapter{Conclusão}%
\label{chap:conclusao}

\section{Considerações Finais}

Nesta monografia foi apresentada a implementação de um \textbf{Processador de Consultas (Query Processor)} como um componente integrante ao DfAnalyzer, instanciando a arquitetura ARMFUL para a análise de simulações computacionais e fluxo de dados em larga escala. Foram abordados os diversos tipos de rastros de proveniência, gerência de fluxo de dados e mapeamento de atributos de dados comumente utilizados em consultas a simulações científicas, assim como os tipos de proveniência prospectiva e retrospectiva.

No \autoref{chap:experimentos}, diversas consultas relacionadas a uma aplicação real~\cite{silva2016situ} foram executadas, ilustrando a utilização do Query Processor.

\perrotta{TODO: Considerações Finais. Depende dos resultados dos experimentos do capítulo 06.}

\section{Trabalhos Futuros}

Dentre algumas ideias de trabalhos futuros para o projeto do processador de consultas, destacam-se:

\begin{itemize}
    % Projeto da Débora: visualizador
    \item a implementação e integração de uma \textbf{interface gráfica} para o processador de consultas, com o intuito de criar uma boa experiência de usabilidade para o usuário e, mais especificamente: (\(i\)) auxiliá-lo a criar e especificar consultas, através da ação de arrastar e soltar com o \textit{mouse} etiquetas com as transformações e conjuntos de dados; e (\(ii\)) apresentá-lo os resultados da execução da consulta em SQL no banco de dados de forma visualmente organizada e limpa.
    % Otimizações de consultas que discutimos.
    \item aumentar a \textbf{performance das consultas} em SQL, com uma (\(i\)) diminuição do \textit{overhead} do processador de consultas, através do aumento do desempenho da conversão das especificações do usuário para a linguagem SQL; e da (\(ii\)) otimização da consulta em si, por exemplo, através de melhorias nas técnicas de indexação, ou da diminuição do número de projeções incluídas na mesma, ou mesmo via um melhor aproveitamento das informações e dados de proveniência disponíveis no banco de dados.
    \item implementar uma forma de \textbf{validação nas especificações do usuário}, permitindo apenas a geração de consultas sintática e semanticamente válidas, com o intuito de melhorar o \textit{design} de interação e diminuir a probabilidade de erros por parte do mesmo, \textit{e.g.}, permitir apenas conjuntos de dados existentes nas projeções especificadas pelo usuário.
    \item permitir a inclusão de subconsultas na consulta principal e adicionar suporte a mais cláusulas da linguagem SQL, a fim de suportar consultas mais complexas: por exemplo, \texttt{GROUP BY}, \texttt{SELECT DISTINCT} e \texttt{LIMIT}.
\end{itemize}
 % Conclusion

  \backmatter{}
  \appendix
  % !TEX encoding = UTF-8 Unicode
% -*- coding: UTF-8; -*-
% vim: set fenc=utf-8

\chapter{Funções auxiliares}%
\label{app:funcoes-auxiliares}

Nesse capítulo são listadas as funções auxiliares referenciadas e utilizadas nos algoritmos da \autoref{sec:algoritmos-utilizados}, na ordem de aparecimento neste documento.

\begin{lstlisting}[language=pseudocode,label={lst:get-next-transformations},caption={[Obtenção das próximas transformações de dados de uma transformação]Obtenção das próximas transformações de dados de uma transformação de dados.}]
function getNextTransformations(D, dt):
    transformations %\leftarrow% {}
    for each %\phi% %\in% %D.\Phi% do:
        if %$\phi.dt_{\textup{previous}}$% = dt and %$\phi.dt_{\textup{next}}$% %\ne% %\varnothing% then:
            transformations %\leftarrow% transformations + 
        end if
    end do
    return transformations
end function
\end{lstlisting}
 \begin{lstlisting}[language=pseudocode,label={lst:has-many-output-datasets},caption={[Contagem dos conjuntos de dados de saída de uma transformação]Contagem dos conjuntos de dados de saída de uma transformação de dados. Retorna verdadeiro caso essa quantidade seja maior do que um, e falso caso contrário.}]
function hasManyOutputDatasets(D, dt):
    count %\leftarrow% 0
    for each %\phi% %\in% %D.\Phi% do:
        if %$\phi.dt_{\textup{previous}}$% = dt and %$\phi.ds$% %\ne% %\varnothing% then:
            count %\leftarrow% count + 1
            if count > 1 then:
                return true
            end if
        end if
    end do
    return false
end function
\end{lstlisting}
 \begin{lstlisting}[language=pseudocode,label={lst:has-many-next-transformations},caption={[Contagem das próximas transformações de dados de uma transformação]Contagem das próximas transformações de dados de uma transformação de dados. Retorna verdadeiro caso essa quantidade seja maior do que um, e falso caso contrário.}]
function hasManyNextTransformations(D, dt):
    count %\leftarrow% 0
    for each %\phi% %\in% %D.\Phi% do:
        if %$\phi.dt_{\textup{previous}}$% = dt and %$\phi.dt_{\textup{next}}$% %\ne% %\varnothing% then:
            count %\leftarrow% count + 1
            if count > 1 then:
                return true
            end if
        end if
    end do
    return false
end function
\end{lstlisting}
 \begin{lstlisting}[language=pseudocode,label={lst:any-transformation-has-many-input-datasets},caption={[Determinação de se pelo menos uma transformação possui mais de um conjunto de dados de entrada.]Determinação de se pelo menos uma transformação de dados possui mais de um conjunto de dados de entrada. Retorna verdadeiro caso positivo, e falso caso contrário.}]
function anyTransformationHasManyInputDatasets(D, dts):
    for each dt %\in% dts do:
        if hasManyInputDatasets(D, dt) then:%~\quad%#%~\autoref{lst:has-many-input-datasets}%
            return true
        end if
    end do
    return false
end function
\end{lstlisting}
 \begin{lstlisting}[language=pseudocode,label={lst:has-many-input-datasets},caption={[Contagem dos conjuntos de dados anteriores a uma transformação]Contagem dos conjuntos de dados anteriores a uma transformação de dados. Retorna verdadeiro caso essa quantidade seja maior do que um, e falso caso contrário.}]
function hasManyInputDatasets(D, dt):
    count %\leftarrow% 0
    for each %\phi% %\in% %D.\Phi% do:
        if %$\phi.dt_{\textup{next}}$% = dt and %$\phi.ds$% %\ne% %\varnothing% then:
            count %\leftarrow% count + 1
            if count > 1 then:
                return true
            end if
        end if
    end do
    return false
end function
\end{lstlisting}
 \begin{lstlisting}[language=pseudocode,label={lst:get-transformation-track},caption={[Obtenção da trilha de transformações de uma transformação.]Obtenção da trilha de transformações de uma transformação.}]
function getTransformationTrack(dtTracks,dt):
    for dtTrack %\in% dtTracks:
        if dt %\in% dtTrack then:
            return dtTrack
        end if
    end do
    return %\varnothing%
end function
\end{lstlisting}

\begin{lstlisting}[language=pseudocode,label={lst:get-previous-transformations},caption={[Obtenção das transformações de dados anteriores a uma transformação]Obtenção das transformações de dados anteriores a uma transformação de dados.}]
function getPreviousTransformations(D, dt):
    transformations %\leftarrow% {}
    for each %\phi% %\in% %D.\Phi% do:
        if %$\phi.dt_{\textup{next}}$% = dt and %$\phi.dt_{\textup{previous}}$% %\ne% %\varnothing% then:
            transformations %\leftarrow% transformations + 
        end if
    end do
    return transformations
end function
\end{lstlisting}

\begin{lstlisting}[language=pseudocode,label={lst:get-next-datasets},caption={[Obtenção dos próximos conjuntos de dados de uma transformação]Obtenção dos próximos conjuntos de dados de uma transformação de dados.}]
function getNextDatasets(D, dt):
    datasets %\leftarrow% {}
    for each %\phi% %\in% %D.\Phi% do:
        if %$\phi.dt_{\textup{previous}}$% = dt and %\phi.ds% %\ne% %\varnothing% then:
            datasets %\leftarrow% datasets + 
        end if
    end do
    return datasets
end function
\end{lstlisting}

\begin{lstlisting}[language=pseudocode,label={lst:get-previous-datasets},caption={[Obtenção dos conjuntos de dados anteriores a uma transformação]Obtenção dos conjuntos de dados anteriores a uma transformação de dados.}]
function getPreviousDatasets(D, dt):
    datasets %\leftarrow% {}
    for each %\phi% %\in% %D.\Phi% do:
        if %$\phi.dt_{\textup{next}}$% = dt and %\phi.ds% %\ne% %\varnothing% then:
            datasets %\leftarrow% datasets + 
        end if
    end do
    return datasets
end function
\end{lstlisting}

\begin{lstlisting}[language=pseudocode,label={lst:get-next-datasets-2},caption={[Obtenção dos próximos conjuntos de dados de um conjunto de dados]Obtenção dos próximos conjuntos de dados de um conjunto de dados.}]
function getNextDatasets(D, ds):
    datasets %\leftarrow% {}
    for each %\phi% %\in% %D.\Phi% do:
        if %$\phi.ds$% = ds:
            datasets %\leftarrow% datasets + {getNextDatasets(%$\phi.dt_{\textup{next}}$%)}%~\quad%#%~\autoref{lst:get-next-datasets}%
        end if
    end do
    return datasets
end function
\end{lstlisting}

\begin{lstlisting}[language=pseudocode,label={lst:get-transformation},caption={[Obtenção da transformação entre dois conjuntos de dados]Obtenção da transformação de dados associada a dois conjuntos de dados.}]
function getTransformation(D, %$ds_1$%, %$ds_2$%):
    for each %\phi% %\in% %D.\Phi% do:
        if %$\phi.ds$% = %$ds_1$% and %$ds_2$% %\in% getNextDatasets(%$\phi.dt_{\textup{next}}$%) then:%~\quad%#%~\autoref{lst:get-next-datasets}%
            return %$\phi.dt_{\textup{next}}$%
        else if %$\phi.ds$% = %$ds_2$% and %$ds_1$% %\in% getNextDatasets(%$\phi.dt_{\textup{next}}$%) then:%~\quad%#%~\autoref{lst:get-next-datasets}%
            return %$\phi.dt_{\textup{next}}$%
        end if
    end do
    return %\varnothing%
end function
\end{lstlisting}
  
  \bibliographystyle{coppetex/coppe-unsrt}
  \bibliography{thesis} % Pro-Tip: use JabRef to manage this file

\end{document}

% Documentos necessários para a defesa do projeto final:
%
%   * Formulário de RCS para o registro de projeto de graduação no BOA:
%
%   - 2017.1 - 2017.2
%   - NACAD - CT
%   - Análise do Rastro de Proveniência em Simulações Computacionais em Larga Escala
%   - Marta Lima de Queirós Mattoso
%
%   * Proposta de Projeto de Graduação:
% 
%   - Marta Lima de Queirós Mattoso
%   - Vítor Silva Sousa
%   - Análise do Rastro de Proveniência em Simulações Computacionais em Larga Escala
%   - Objetivo: Desenvolvimento do Processador de Consultas para análise dos rastros de proveniência em simulações computacionais.
%   - Metodologia: Desenvolvimento de um sistema de Processamento de Consultas em Java