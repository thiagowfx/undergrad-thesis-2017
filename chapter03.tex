% !TEX encoding = UTF-8 Unicode
% -*- coding: UTF-8; -*-
% vim: set fenc=utf-8

\chapter{Trabalhos Relacionados}%
\label{chap:trabalhos-relacionados}

Tipos de consultas 1, 2, 3

\perrotta{Capítulo 03 --- queries do tipo 1, 2, 3}

\section{Dados científicos isolados} % TODO: better name

Acesso e extração de dados --> uso de programas específicos para cada formato de arquivo

Técnicas de indexação --> FastBit, NoDB, RAW

\perrotta{Capítulo 03 --- Soluções para análise de dados científicos isolados (parsing; extratores ou indexações)}

\section{Fluxos de arquivos} % TODO: better name
\perrotta{Capítulo 03 --- Soluções para análise do fluxo de arquivos (citar sistemas existentes: Chiron; Pegasus; Kepler; etc)}

sistema de workflows --> gerência do fluxo de arquivos

\section{Fluxo de elementos de dados em múltiplos arquivos}
\perrotta{Capítulo 03 --- Soluções para análise do fluxo de elementos de dados relacionados em múltiplos arquivos
DfAnalyzer (grupo)
YesWorkflow
NoWorkflow
CCPE (mais sistemas)
}
