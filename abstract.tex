% !TEX encoding = UTF-8 Unicode
% -*- coding: UTF-8; -*-
% vim: set fenc=utf-8

\begin{abstract}

%REVIEW: resumo em português
Simulações computacionais potencialmente recebem e geram uma enorme quantidade de dados e arquivos brutos %silva2015analyzing
apresentados em diversos formatos padrões estabelecidos pelo domínio da aplicação, %silva2017raw
tal como o \textit{HDF5} em simulações de dinâmica de fluidos. %silva2017raw
Durante essas simulações, usuários em geral precisam rastrear grandezas de interesse (como resíduos, tempo de execução e estimativas de erros) %silva2016situ
e dados de domínio com base em elementos relacionados de diversos arquivos, a fim de controlar sua execução ao máximo possível; %silva2016situ
no entanto, esse rastreio é realizado usualmente somente ao fim da simulação. %silva2016situ
Apesar dos formatos comuns de arquivo serem suportados por diversas linguagens de programação e bibliotecas, é necessário desenvolver programas para a análise em grande escala dos mesmos, %silva2015analyzing
o que é custoso se realizado somente com bancos de dados, pois eles exigem que os dados brutos sejam estruturados e carregados em memória. %silva2015analyzing
No entanto, o gerenciamento dessas simulações com sistemas de gerenciamento de simulação científica (\textit{SWfMS}) com ciência do fluxo de dados cria a habilidade de tirar proveito de dados de proveniência da mesma para relacionar e analisar os dados brutos da aplicação. %silva2015analyzing
Esse trabalho apresenta uma solução baseada em uma arquitetura de componentes (ARMFUL) que utiliza proveniência de dados para extrair e relacionar grandezas de interesse para a realização de consultas \textit{online} (\textit{in situ}). %silva2016situ

\end{abstract}

\begin{foreignabstract}

%REVIEW: resumo em inglês
Computer simulations potentially receive and produce lots of raw data files.
These files usually follow a \textit{de facto} standard data format established by the application domain,
such as HDF5 in fluid dynamics simulations.
In these simulations, users often need to track quantities of interest (residuals, execution times, error estimates) in order to control as much as possible its execution.
However, this tracking is usually done only once the simulation ends.
This work presents a solution based on provenance data with the purpose of online (\textit{in situ}) extraction and connection among quantities of interest.

\end{foreignabstract}
