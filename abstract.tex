% !TEX encoding = UTF-8 Unicode
% -*- coding: UTF-8; -*-
% vim: set fenc=utf-8

\begin{abstract}

%REVIEW: resumo em português
Simulações computacionais potencialmente recebem e geram uma enorme quantidade de dados e arquivos brutos. %[1]
Esses arquivos usualmente seguem um formato padrão estabelecido pelo domínio da aplicação, %[1]
tal como o HDF5 em simulações de dinâmica de fluidos. %[2]
Nessas simulações, usuários geralmente precisam rastrear grandezas de interesse (resíduos, tempo de execução, estimativas de erros) a fim de controlar sua execução ao máximo possível; %[3]
no entanto, esse rastreio é realizado usualmente somente ao fim da simulação. %[3]
Esse trabalho apresenta uma solução baseada em dados de proveniência para extrair e relacionar grandezas para consultas \textit{online} (in situ).

\end{abstract}

\begin{foreignabstract}

%REVIEW: resumo em inglês
Computer simulations potentially receive and produce lots of raw data files. 
These files usually follow a standard data format established by the application domain,
such as HDF5 in fluid dynamics simulations.
In these simulations, users often need to track quantities of interest (residuals, execution times, error estimates) in order to control as much as possible its execution.
However, this tracking is usually done only once the simulation ends.
This work presents a solution based on provenance data with the purpose of online (\textit{in situ}) extraction and connection among quantities of interest.

\end{foreignabstract}
