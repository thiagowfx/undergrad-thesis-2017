% !TEX encoding = UTF-8 Unicode
% -*- coding: UTF-8; -*-
% vim: set fenc=utf-8

\begin{abstract}

%REVIEW: resumo em português
Simulações computacionais frequentemente consomem e produzem um grande volume de dados. Parte desses dados é armazenado em arquivos brutos %silva2015analyzing
, que podem assumir diversos formatos, de acordo com o domínio da aplicação, %silva2017raw
tal como o \textit{HDF5} em simulações de dinâmica de fluidos computacionais. %silva2017raw
Durante essas simulações, usuários em geral precisam rastrear grandezas de interesse (como resíduos, tempo de execução e estimativas de erros) %silva2016situ
com base em elementos de dados relacionados de diversos arquivos, a fim de controlar sua execução ao máximo possível %silva2016situ.
No entanto, esse rastreio é realizado usualmente somente ao fim da simulação. %silva2016situ
Apesar dos formatos comuns de arquivo serem apoiados por diversas linguagens de programação e bibliotecas,
os usuários comumente precisam desenvolver programas \textit{ad-hoc} para a análise em grande escala dos mesmos %silva2015analyzing,
que pode ser ainda mais custoso se realizado com bancos de dados, pois eles exigem que os dados \silva[brutos]{ao invés de bruto, utilizar dado científico}
sejam estruturados e carregados em memória. %silva2015analyzing
Nesse cenário, Sistemas de Gerência de Workflows Científicos (\textit{SGWfC}) com ciência do fluxo de dados têm empregado a abstração de workflows científicos
para apoiar a execução paralela de simulações computacionais, em que os dados de proveniência podem favorecer a gerência de elementos de dados
relacionados em múltiplos arquivos de dados científicos (ou seja, produzidos por diferentes programas de simulação) %silva2015analyzing.
Esse trabalho utiliza a arquitetura de componentes ARMFUL, que utiliza uma abordagem baseada em dados de proveniência para extrair e
relacionar grandezas de interesse. Mais especificamente, esse trabalho contribui com um mecanismo de processamento de consultas durante a execução de simulações computacionais,
considerando-se o fluxo de elementos de dados ao longo das simulações computacionais.

\end{abstract}

\begin{foreignabstract}

%REVIEW: resumo em inglês
\silva[Computer simulations potentially receive and produce lots of raw data files.
These files usually follow a \textit{de facto} standard data format established by the application domain,
such as HDF5 in fluid dynamics simulations.
In these simulations, users often need to track quantities of interest (residuals, execution times, error estimates) in order to control as much as possible its execution.
However, this tracking is usually done only once the simulation ends.
This work presents a solution based on provenance data with the purpose of online (\textit{in situ}) extraction and connection among quantities of interest.]{traduzir novamente}

\end{foreignabstract}
