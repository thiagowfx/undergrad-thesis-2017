% !TEX encoding = UTF-8 Unicode
% -*- coding: UTF-8; -*-
% vim: set fenc=utf-8

\begin{abstract}

Simulações computacionais frequentemente consomem e produzem grandes volumes de dados, parte dos quais é armazenada em arquivos brutos em 
diversos formatos de acordo com o domínio da aplicação, tal como o \textit{HDF5} em simulações de dinâmica de fluidos computacionais. Durante essas simulações, usuários em geral precisam rastrear grandezas de interesse com base em elementos de dados relacionados de vários arquivos, a fim de otimizar o controle de sua execução. No entanto, esse rastreamento costuma ser feito somente ao fim da simulação. Apesar dos formatos de arquivo serem apoiados por diversas bibliotecas, os usuários em geral precisam desenvolver programas \textit{ad-hoc} para realizarem uma análise em grande escala, a qual pode ser ainda mais custosa se realizada com bancos de dados, pois eles exigem que os dados científicos sejam estruturados e carregados. Nesse cenário, Sistemas de Gerência de Workflows Científicos (\textit{SGWfC}) com ciência do fluxo de dados têm empregado a abstração de \textit{workflows} científicos para apoiar a execução paralela dessas simulações, em que os dados de proveniência podem favorecer a gerência de elementos de dados relacionados em múltiplos arquivos de dados científicos (\textit{i.e.}, produzidos por diferentes programas de simulação). Este trabalho utiliza a arquitetura de componentes ARMFUL, que baseia-se em dados de proveniência para extrair e relacionar grandezas de interesse. Mais especificamente, contribui com um mecanismo de processamento de consultas durante a execução de simulações computacionais, considerando-se o fluxo de elementos de dados ao longo das mesmas. 

\end{abstract}

\begin{foreignabstract}

Computer simulations frequently consume and produce lots of raw data files, which usually follow a \textit{de facto} standard data format established by the application domain, such as \textit{HDF5} in fluid dynamics simulations.
During these simulations, users often need to track quantities of interest based on elements of related data from several raw files, in order to control the execution as much as possible. Nonetheless, this tracking is usually done only once the simulation ends.
Notwithstanding the file formats being supported by several programming languages and libraries, users generally need to develop ad-hoc programs to perform a specific analysis in large scale, which can be even more costly if performed with database systems, because they require the raw data files to be structured and loaded. In this scenario, dataflow-aware Scientific Workflow Management Systems (SWfMS) have resorted to scientific workflow abstractions to support parallel execution of these simulations, in which provenance data can help to manage related data elements in multiple raw data files (\textit{i.e.}, produced by different simulation programs).
This work uses the component-based ARMFUL architecture, which is based on provenance data to extract and relate quantities of interest. More precisely, it contributes with a query processing mechanism for computer simulations, considering the flow of data elements during its execution.

\end{foreignabstract}
