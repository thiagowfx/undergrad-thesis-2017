% !TEX encoding = UTF-8 Unicode
% -*- coding: UTF-8; -*-
% vim: set fenc=utf-8

\chapter{Arquitetura ARMFUL}%
\label{chap:arquitetura-armful}

Nesse capítulo será apresentada a arquitetura ARMFUL, um trabalho introduzido recentemente em~\cite{silva2016situ,silva2017raw} pelo laboratório de \abbrev{NACAD}{Núcleo Avançado de Computação de Alto Desempenho} Núcleo Avançado de Computação de Alto Desempenho (NACAD) da \abbrev{UFRJ}{Universidade Federal do Rio de Janeiro} UFRJ (Universidade Federal do Rio de Janeiro).

\section{Visão geral}

% DfAnalyzer (instância da ARMFUL) --> conjunto de componentes
% Provenance Gatherer (origem da proveniência para o banco de dados que eu vou utilizar na QueryProcessor)
% QueryProcessorCaptura de dados de proveniência}

\subsection{Extração e indexação de dados científicos}

% query tipo 1 --> in situ

\subsection{Ingestão de dados}

% prov df (alimentar a base de dados com informacoes de proveniencia)
% ProvenanceGatherer

\subsection{Processamento de consultas}

% QueryProcessor

% melhor traduzir tudo, exceto DfAnalyzer

\section{DfAnalyzer: uma instanciação da arquitetura ARMFUL}

\subsection{Provenance Data Gatherer (PDG)}

\subsection{Raw Data Extractor (RDE)}

\subsection{Raw Data Indexer (RDI)}

\subsection{Query Processor (QP)}